\chapter{File System(FS)}
\label{ch:FS}
\section{Files}
It is essential that a computer system has a facility for storing data over longer periods of time
and for retrieving the stored data. Such a facility is called a FS. Evidently, a FS cannot accommodate
all possible data types and structures that will be programmed in the future.  Hence, it is necessary
to provide a simple, yet flexible enough base structure that allows any data structure to be mapped
onto this base structure (and vice-versa) in a reasonably straight-forward and efficient way. This base
structure, called \emph{file}, is \emph{a sequence of bytes}. As a consequence, any given structure to
be transformed into a file must be sequentialized. The notion of sequence is indeed fundamental, and it
requires no further explanation and theory. We recall that texts are sequences of characters, and that
characters are typically represented as bytes.

The sequence is also the natural abstraction of all physically moving storage media. Among them
are magnetic tapes and disks. Magnetic media have the welcome property of non-volatility and
are therefore the primary choices for storing data over longer periods of time, especially over
periods where the equipment is switched off. Sequential access is also necessary for media that
allow access only by large blocks, such as flash-RAMs and SD-cards.

A further advantage of the sequence is that its transmission between media is simple too. The
reason is that its structural information is inherent and need not be encoded and transmitted in
addition to the actual data. This implicitness of structural information is particularly convenient in
the case of moving storage media, because they impose strict timing constraints on transmission
of consecutive elements. Therefore, the process which generates (or consumes) the data must
be effectively decoupled from the transmission process that observes the timing constraints. In
the case of sequences, this decoupling is simple to achieve by dividing a sequence into
subsequences which are buffered. A sequence is output to the storage medium by alternately
generating data (and filling the buffer holding the current subsequence) and transmitting data
(fetching elements from the buffer and transmitting them). The size of the subsequences (and the
buffer) depends on the storage medium under consideration: there must be no timing constraints
between accesses to consecutive subsequences.

The file is not a static data structure like the array or the record, because the length may increase
dynamically, i.e. during program execution. On the other hand, the sequence is less flexible than
general dynamic structures, because it cannot change its form, but only its length, since elements
can only be appended but not inserted. It might therefore be called a \emph{semi-dynamic structure}.
The discipline of purely sequential access to a file is enforced by restricting access to calls of
specific procedures, typically read and write procedures for scanning and generating a file. In the
jargon of data processing, a file must be opened before reading or writing is possible. The
opening implies the initialization of a reading and writing mechanism, and in particular the fixing of
its initial position. Hence each (opened) file not only has a value and a length, but also a position
attributed to it. If reading must occur from several positions (still sequentially) alternately, the
file is "multiply opened"; it implies that the same file is represented by several variables, each
denoting a different position.

This widespread view of files is conceptually unappealing, and the Oberon FS therefore departs from it
by introducing the notion of a rider. A file simply has a value, the sequence of bytes, and a length,
the number of bytes in the sequence. Reading and writing occurs through a \emph{rider}, which
\emph{denotes a position}. "Multiple opening" is achieved by simply using several riders riding on the
same file. Thereby the 2 concepts of data structure (file) and access mechanism (rider) are clearly
distinct and properly disentangled.

Given a file \verb|f|, a rider \verb|r| is placed on a file by the call \verb|Files.Set (r, f, pos)|,
where \verb|pos| indicates the position from which reading or writing is to start. Calls of
\verb|Files.Read (r, x)| and \verb|Files.Write (r, x)| implicitly increment the position beyond the
element read or written, and the file is implicitly denoted via the explicit parameter \verb|r|, which
denotes a rider. The rider has 2 (visible) attributes, namely \verb|r.eof| and \verb|r.res|. The former
is set to FALSE by \verb|Files.Set|, and to TRUE when a read operation could not be performed, because
the end of the file had been reached. \verb|r.res| serves as a result variable in procedures
\verb|ReadBytes| and \verb|WriteBytes| allowing one to check for correct termination.

A FS must not only provide the concept of a sequence with its accessing mechanism, but also a registry.
This implies that files be identified, that they can be given a name by which they are registered and
retrieved. The registry or collection of registered names is called the file system's \emph{directory}.
Here we wish to emphasize that the concepts of files as data structure with associated access facilities
on the one hand, and the concept of file naming and directory management on the other hand must also be
considered separately and as independent notions.  In fact, in the Oberon their implementation
underscores this separation by the existence of 2 modules: \verb|Files| and \verb|FileDir|. The
following procedures are available. They are summarized by the interface specification (definition) of
module \verb|Files|.
\begin{verbatim}
DEFINITION Files;
  TYPE File = POINTER TO FileDesc;
   FileDesc = RECORD END ;
      Rider = RECORD eof: BOOL; res: INT END ;

  PROC Old(name: ARRAY OF CHAR): File;
  PROC New(name: ARRAY OF CHAR): File;
  PROC Register(f: File);
  PROC Close(f: File);
  PROC Purge(f: File);
  PROC Length(f: File): INT;
  PROC Date(f: File): INT);

  PROC Set(VAR r: Rider; f: File; pos: INT);
  PROC ReadByte(VAR r: Rider; VAR x: BYTE);
  PROC ReadBytes(VAR r: Rider; VAR x: ARRAY OF BYTE; n: INT);
  PROC Read(VAR r: Rider; VAR ch: CHAR);
  PROC ReadInt(VAR r: Rider; VAR n: INT);
  PROC ReadSet(VAR r: Rider; VAR s: SET);
  PROC ReadReal(VAR r: Rider; VAR x: REAL);
  PROC ReadString(VAR r: Rider; VAR s: ARRAY OF CHAR);
  PROC ReadNum(VAR r: Rider; VAR n: INT);

  PROC WriteByte(VAR r: Rider; x: BYTE);
  PROC WriteBytes(VAR r: Rider; x: ARRAY OF BYTE; n: INT);
  PROC WriteInt(VAR r: Rider; n: INT);
  PROC WriteSet(VAR r: Rider; s: SET);
  PROC WriteReal(VAR r: Rider; x: REAL);
  PROC WriteString(VAR r: Rider; x: ARRAY OF CHAR);
  PROC WriteNum(VAR r: Rider; n: INT);
  PROC Pos(VAR r: Rider): INT;
  PROC Base(VAR r: Rider): File;

  PROC Rename(old, new: ARRAY OF CHAR; VAR res: INT);
  PROC Delete(name: ARRAY OF CHAR; VAR res: INT);
END Files.
\end{verbatim}

\verb|New(name)| yields a new (empty) file without registering it in the directory. \verb|Old(name)|
retrieves the file with the specified name, or yields \verb|NIL|, if it is not found in the directory.
\verb|Register(f)| inserts the name of \verb|f| (specified in the call of \verb|New|) in the directory.
An already existing entry with this name is replaced. \verb|Close(f)| must be called after writing is
completed and the file is not to be registered. \verb|Close| actually stands for "close buffers", and
is implied in the procedure \verb|Register|.  Procedure \verb|Purge| will be explained at the end of
\S \ref{sec:randstore}.

The sequential scan of a file \verb|f| (reading characters) is programmed as shown in the following
template:
\begin{verbatim}
  VAR f: Files.File; r: Files.Rider;
  
  f := Files.Old(name);
  IF f # NIL THEN
    Files.Set (r, f, 0); Files.Read (r, x);
    WHILE ~ r.eof DO ... x ...; Files.Read(r, x) END
  END
\end{verbatim}

The analogous template for a purely sequential writing is:
\begin{verbatim}
  f := Files.New(name); Files.Set(r, f, 0);
  WHILE ... DO Files.Write (r, x); ... END
  Files.Register(f)
\end{verbatim}

There exist 2 further procedures; they do not change any files, but only affect the directory. \\
\verb|Delete(name, res)| causes the removal of the named entry from the directory. \\
\verb|Rename(old, new, res)| causes the replacement of the directory entry old by new.

It may surprise the reader that these 2 procedures, which affect the directory only, are exported
from module \verb|Files| instead of \verb|FileDir|. The reason is that the presence of the 2 modules,
together forming the FS, is also used for separating the interface into a public and a private (or
semi-public) part. The definition (in the form of a symbol file) of \verb|FileDir| is not intended
to be freely available, but restricted to use by system programmers. This allows the export of
certain sensitive data, (such as file headers) and sensitive procedures (such as \verb|Enumerate|)
without the danger of misuse by inadvertent users.

Module \verb|Files| constitutes a most important interface whose stability is utterly essential,
because it is used by almost every module programmed. During the entire time span of development
of the Oberon, this interface had changed only once. We also note that this interface is very terse,
a factor contributing to its stability. Yet, the offered facilities have in practice over years
proved to be both necessary and sufficient.
