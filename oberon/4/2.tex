\section{Viewers as Objects}
\label{sec:viewers}
Although everybody seems to agree on the meaning of the term \emph{viewer},
no 2 different system designers actually do.  The original role of a viewer
as merely a separate display area has meanwhile become heavily overloaded
with additional functionality.  Depending on the underlying system are viewers' individual views
on a certain configuration of objects, carriers of tasks, processes, applications, etc.
Therefore, we first need to define our own precise understanding of the concept of viewer.

The best guide to this aim is the abstract data type \verb|Viewer| we introduced in \ref{ch:task}.
We recapitulate: \verb|Viewer| serves as a template describing viewers abstractly as “black boxes”
in terms of a state of visibility, a rectangle on the display screen, and a message handler.
The exact functional interface provided by a given variant of viewer
is determined by the set of messages accepted.
This set is structured as a customized hierarchy of type extensions.

We can now obtain a more concrete specification of the role of viewer
by identifying some basic categories of universal messages
that are expected to be accepted by all variants of viewer.
For example, we know that messages reporting about user interactions
as well as messages defining a generic operation are universal.
These 2 categories of universal messages document the roles of viewers as interactive tasks
and as parts of an integrated system respectively.
In total, there are 4 such categories.
They are here listed together with the corresponding topic and message dispatchers:
\begin{table}[h!]
  \setlength{\tabcolsep}{3pt}
  \begin{tabular}{r|l|l}
    Dispatcher               &Topic
                             &Message \\\hline
    {\small  task scheduler} &{\small dispatching tasks}
                             &{\small report user interaction }\\
    {\small cmd interpreter} &{\small processing command}
                             &{\small define generic operation}\\
    {\small    view manager} &{\small organizing display area}
                             &{\small location/size change}\\
    {\small     doc manager} &{\small operating on document}
                             &{\small content/format change}
  \end{tabular}
\end{table}
\\These topics essentially define the role of viewers.
In short, we may look at an viewer as a non-overlapped rectangular box on the screen both
\begin{itemize}
  \item[-] acting as an \emph{integrated display area} for some objects of a document, and
  \item[-] \emph{representing an interactive task} in the form of a sensitive editing area.
\end{itemize}

Shifting emphasis a little and regarding the various message dispatchers as subsystems,
we recognize immediately the role of viewers as \emph{integrator of the different subsystems
via message-based interfaces}.  In this light,
\verb|Viewer| appears as a common object-oriented basis of Oberon's subsystems.

The topics listed above constitute some kind of contents backbone of the \ref{ch:task},
\ref{ch:display} and \ref{ch:text}.  Task scheduling and command interpreting
are already familiar to us from \S \ref{sec:scheduler} and \ref{sec:command}.
Viewer and text management will be the topics of \S \ref{sec:dispmanagement}
and \ref{sec:textmanagement}, respectively.  Thereby, the built-in type \verb|Text|
will serve as a prime example of a document type.  The activities that a viewer performs
are basically controlled by events or, more precisely, by messages representing event notices.
We shall explain this in \S \ref{sec:dispmanagement} and \ref{sec:textframes}
in detail cases of an abstract class of standard viewers and
a class of viewers displaying standard text, respectively.

Here is a preliminary overview of some archetypal kinds of message:
\begin{itemize}
  \item After each stroke a keyboard message with the typed character is sent to the focus viewer
    and after each click a mouse message of the new state is sent to the viewer containing the mouse.
  \item The message often representing some generic operation is to be interpreted individually
    by recipients. Obvious examples are "return current textual selection", "copy-over stretch
    of text", and "produce a copy (clone)". Notice that generic operation is the key to extensibility.
  \item In a tiling viewer environment, every opening of a new viewer and every change of size
    or location of an existing viewer has an obvious effect on adjacent viewers. The viewer manager
    therefore issues a message to every affected viewer requesting to adjust its size appropriately.
  \item Whenever the contents or the format of a document has changed, a message notifying
    all visible viewers of the change is broadcast. Notice that broadcasting messages by a model (document)
    to the entirety of its potential views (viewers) is an interesting implementation of the famous
    model-view-controller (MVC) pattern that dispenses models from “knowing” (registering) their views.
\end{itemize}
