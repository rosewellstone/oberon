\section{Standard display configs and toolbox}
\label{sec:stdispconf}
Let us now take up again our earlier topic of configuring the display area. We have seen that
no specific layout of the display area is distinguished by the general viewer management itself.
However, some support of the familiar standard Oberon display look is provided by \verb|Oberon|.

In the terminology of this module, a standard configuration
consists of one or several horizontally adjacent displays,
where a display is a pair consisting of 2 equal height tracks,
\begin{table}[h!]
  \centering
  \begin{tabular}{l}
    a \emph{user track} on the left, and \\
    a \emph{system track} on the right.
  \end{tabular}
\end{table}
\\Note that even though no reference to any physical monitor is made,
a display is typically associated with a monitor in reality.
This is the relevant excerpt of the definition:
\begin{verbatim}
 DEFINITION Oberon;
   PROC OpenDisplay(UW, SW, H: INT);
   PROC OpenTrack(X, W: INT);
   PROC DisplayWidth (X: INT): INT;
   PROC DisplayHeight(X: INT): INT;
   PROC UserTrack    (X: INT): INT;
   PROC SystemTrack  (X: INT): INT;
   PROC AllocateUserViewer  (DX: INT; VAR X,Y: INT);
   PROC AllocateSystemViewer(DX: INT; VAR X,Y: INT);
 END Oberon.
\end{verbatim}
\begin{itemize}
  \item \verb|OpenDisplay| initializes and opens a new display of
    \begin{table}[h!]
      \centering
      \begin{tabular}{r r}
	parameter & dimension \\\hline
        \verb|UW| &   User track Width, \\
        \verb|SW| & System track Width, \\
        \verb|H|  & Height.
      \end{tabular}
    \end{table}
  \item \verb|OpenTrack| overlays the sequence of existing tracks spanned by the segment
    \verb|[X, X + W)| by a new track.
  \item Both \verb|OpenDisplay| and \verb|OpenTrack| take from the client
    the burden of creating a filler viewer.
  \item \verb|DisplayWidth|, \verb|DisplayHeight|, \verb|UserTrack| and \verb|SystemTrack|
    return width or height of the respective structural entity located at position \verb|X|
    in the display area.
  \item \verb|AllocateUserViewer| and \verb|AllocateSystemViewer|
    make proposals for the allocation of a new viewer
    in the desired track of the display located at \verb|DX|.
\end{itemize}
In 1st priority, the location is determined by the system pointer that can be set manually.
If the pointer is not set, a location is calculated on the basis of some heuristics
whose strategies rely on different splitting fractions
that are applied in the user track and in the system track respectively,
with the aim of generating aesthetically satisfactory layouts.

In addition to the programming interface provided by \verb|Oberon|
for the case of standard display layouts,
the display management section in the \verb|System| toolbox provides a user interface:
\begin{verbatim}
  DEFINITION System; (*Display management*)
    PROC Open;   (*viewer*)
    PROC Close;  (*viewer*)
    PROC CloseTrack;
    PROC Recall; (*most recently closed viewer*)
    PROC Copy;   (*viewer*)
    PROC Grow;   (*viewer*)
    PROC Clear;  (*system log*)
  END System.
\end{verbatim}
In turn, these commands are called to
\begin{table}[h!]
  \centering
  \begin{tabular}{l}
    open a text viewer in the system track, \\
    close a viewer, \\
    close a track, \\
    recall (and reopen) the most recently closed viewer, \\
    copy a viewer, \\
    grow a viewer, and \\
    clear the system log.
  \end{tabular}
\end{table}
\\\verb|Close|, \verb|Copy|, \verb|Grow|, \verb|CloseTrack|, and \verb|Recall| are generic commands.
The 1st 3 are typically included in the title bar of a menu viewer.
Their detailed implementations follow subsequently.
