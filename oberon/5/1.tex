\section{Text as an abstract data type}
\label{sec:textyp}
The concept of abstraction is arguably the most important achievement of PL development.
It provides a powerful tool to create simplified views of complicated things and connections.
2 prominent examples of program abstractions
\begin{table}[h!]
  \centering
  \begin{tabular}{r l}
                                     & \small{embodying simplified views on} \\\hline
    \small{abstract data types}      & \small{a certain kind of data, and} \\
    \small{definitions (interfaces)} & \small{a certain piece of program.}
  \end{tabular}
\end{table}

We shall now give a precise definition of the notion of text in Oberon
by presenting it as an abstract data type.
It is important not to confuse this type with the far less powerful one \verb|String|
as it is often supported by advanced PLs.
Here we carefully avoid revealing any implementation aspects of the abstract type \verb|Text|.
Our viewpoint is that of an application program operating on text abstractly
or using it as a medium of communication.

Nevertheless, let us first use a symbolic looking glass to get a refined understanding
of the concept of character in the context of rich texts.
We know that each character represents a textual element of information.
If displayed, it also refers to some specific graphical pattern, often called \emph{glyph}.
In Oberon, we do justice to both aspects by thinking of the ASCII as an index into a font
that is into a set of glyphs of the same style.
Representing characters as pairs \verb|(font, ref)|,
where \verb|font| designates a font and \verb|ref| the character's ASCII code
and adding 2 more attributes \emph{color} and \emph{vertical offset},
we get to a quadruple representation \verb|(font, ref, col, voff)| of characters.
The components font, color, and vertical offset together are often referred to as \emph{looks}.
With that, we can now define a (rich) text as a \emph{sequence of characters with looks}.
We shall treat the topic of fonts and glyphs thoroughly in \ref{sec:fontmachinery}.

For the moment, however, let us continue our discussion of the abstract data type \verb|Text|.
Formally, we define it as
\begin{verbatim}
  Text = POINTER TO TextDesc;
  TextDesc = RECORD len: INT; notify: Notifier END;
\end{verbatim}
There is only 1
\begin{itemize}
  \item state variable \verb|len|, and

    Represents the current length of the described text
    (i.e. the number of characters in the sequence).
  \item method \verb|notify|.

    Occasionally called \emph{after-method}.
    Notify interested clients of state changes.
\end{itemize}

By definition, each abstract data type comes with a complete set of operations.
In the case of \verb|Text|, 3 groups corresponding to 3 different topics need to be considered,
\begin{enumerate}
  \item loading (from file),
        storing (to file),
  \item editing, and
  \item accessing (reading and writing) respectively.
\end{enumerate}

\subsection{Loading and Storing Text}
The $1^{st}$ file group introduces a pair of mutually inverse operations,
\emph{internalize} and \emph{externalize}, respectively meaning
\begin{itemize}
  \item load from file and build up an internal data structure, and
  \item serialize the internal data structure and store it in file.
\end{itemize}
There are 3 corresponding procedures:
\begin{verbatim}
  PROC Open (T: Text; name: ARRAY OF CHAR);
  PROC Load (T: Text; f: Files.File; pos: INT;
                                 VAR len: INT);
  PROC Store(T: Text; f: Files.File; pos: INT;
                                 VAR len: INT);
\end{verbatim}
Logical entities like texts are stored on external media in form of \emph{section},
addressed by pair \verb|(file, pos)| consisting of
\begin{table}[h!]
  \centering
  \begin{tabular}{l}
    \verb|file| \textcolor{gray}{descriptor, and
    starting} \verb|pos|\textcolor{gray}{ition}.
  \end{tabular}
\end{table}
\\In general, the structure of section obeys:
\begin{verbatim}
  section = identification type length contents.
\end{verbatim}

\begin{table}[h!]
  \begin{tabular}{l l}
    \verb|Open|  & internalizes a named text file \\
                 & (consisting of a single text section) \\
    \verb|Load|  & internalizes an arbitrary text section \\
                 & starting at \verb|(f, pos)| \\
    \verb|Store| & externalizes a text section to \verb|(f, pos)| \\
    \verb|T|     & designates the internalized text \\
    \verb|len|   & returns the length of the section
  \end{tabular}
\end{table}
Note that in case of \verb|Load| the identification of section
must have been read and consumed before the loader is called.

\subsection{Editing Text}
Next group operations support text editing, comprising:
\begin{verbatim}
  PROC Append     (T: Text;           B: Buffer);
  PROC Insert     (T: Text; pos: INT; B: Buffer);
  PROC Delete     (T: Text; beg, end: INT);
  PROC ChangeLooks(T: Text; beg, end: INT; sel: SET;
                   fnt: Fonts.Font; col, voff: INT);
\end{verbatim}
Again, we should 1st explain the types of parameters:
\begin{itemize}
  \item \verb|Delete| and \verb|ChangeLooks| each take a stretch of text as an argument which,
    by definition, is an interval \verb|[beg, end)| within the given text.
  \item In the parameter lists of \verb|Insert| and \verb|Append|
    we recognize a new data type \verb|Buffer|.
\end{itemize}
Buffers are a facility to hold anonymous sequences of characters.
\verb|Buffer| presents itself again as an abstract data type:
\begin{verbatim}
  Buffer  = POINTER TO BufDesc;
  BufDesc = RECORD len: INT END;
\end{verbatim}
\verb|len| specifies the current length of the buffered sequence.

The following represent the intrinsic operations on buffers:
\begin{verbatim}
  PROC OpenBuf(B: Buffer);
  PROC Copy(SB, DB: Buffer);
  PROC Save(T: Text; beg, end: INT; B: Buffer);
\end{verbatim}
Their function, in turn:
\begin{itemize}
  \item opening a given buffer \verb|B|,
  \item copying buffer \verb|SB| to \verb|DB|, and
  \item saving a stretch \verb|[beg, end)| of text in a given buffer,
    recalling the most recently deleted stretch of text and putting it into \verb|B|.
\end{itemize}

Buffer is used as an auxiliary data type in editing.
\begin{itemize}
  \item \verb|Delete| d\verb|~|s the given stretch \verb|[beg, end)| within text \verb|T|,
  \item \verb|Insert| i\verb|~|s the buffer's contents at \verb|pos|\textcolor{gray}{ition}
    within \verb|T|, and
  \item \verb|Append(T, B)| is a shorthand of \verb|Insert(T, T.len, B)|.
\end{itemize}
Note that, as a side-effect of \verb|Insert| and \verb|Append|, the buffer involved is emptied.

Finally, \verb|ChangeLooks| allows to change selected looks
within the given stretch \verb|[beg, end)| of \verb|T|.
\verb| sel| is a mask selecting a subset of the looks set
\verb|{f|\textcolor{gray}{o}\verb|nt|, \verb|col|\textcolor{gray}{or},
\verb|v|\textcolor{gray}{ertical} \verb|off|\textcolor{gray}{set}\verb|}|.

It is time now to come back to the notifier concept.
Recapitulate that notify is an \emph{after-method}.
It must be installed by the client when opening the text
and is called at the end of every editing operation.
Its signature
\begin{verbatim}
  Notifier = PROC(T: Text; op, beg, end: INT);
\end{verbatim}
\verb|op|, \verb|beg|, and \verb|end| report
\begin{itemize}
  \item about the \verb|op|\textcolor{gray}{eration} that calls the notifier, and
  \item on the affected stretch \verb|[beg, end)| of the text.
\end{itemize}
There're 3 possible \verb|op|s corresponding to different editing operations:
\begin{table}[h!]
  \centering
  \begin{tabular}{c l}
    \verb|op|     & operation \\\hline
    \verb|delete| & \verb|Delete| \\
    \verb|insert| & \verb|Insert|(, \verb|Append|) \\
    \verb|replace|& \verb|ChangeLooks|
  \end{tabular}
\end{table}

By far the most important application of the notifier is updating the display,
i.e. adjusting all affected views of the text
that are currently displayed to the new state of the text (the model).
We shall come back to this important matter when discussing text frames in \ref{sec:textframes}.

In concluding this part it is worth noting that
the operations just discussed have been designed to be equally useful for
\begin{itemize}
  \item interactive text editors, as well as
  \item programmed text generators/manipulators.
\end{itemize}

\subsection{Accessing Text}
The 3rd and last group of operations on texts:
Accessing, i.e. \emph{reading} and \emph{writing}.

According to the principle of separation of concerns, one of our guiding principles,
the access mechanism operates on extra aggregates
(called \emph{readers} and \emph{writers}) rather than on texts themselves.

Readers are used to read texts sequentially. Their type:
\begin{verbatim}
  Reader = RECORD eot: BOOL; (*end of text*)
             fnt: Fonts.Font; col, voff: INT
           END;
\end{verbatim}

A reader must 1st be opened at the desired position in the text
before it can then be moved forward incrementally by reading character-by-character.
Its state variables indicate end-of-text and expose the looks of the character last read.
The corresponding operators are
\begin{verbatim}
  PROC OpenReader(VAR R: Reader; T: Text; pos: INT);
  PROC Read      (VAR R: Reader; VAR ch: CHAR);
\end{verbatim}
\begin{itemize}
  \item \verb|OpenReader| sets up a reader \verb|R| at position \verb|pos| in text \verb|T|.
  \item \verb|Read| returns the character at the current position of \verb|R|
    and makes \verb|R| move to the next position.
\end{itemize}
The current position of \verb|R| is returned by a call to the function:
\begin{verbatim}
  PROC Pos(VAR R: Reader): INT;
\end{verbatim}

In \ref{ch:task} we learned that commands plus parameter lists are often embedded in ordinary texts.
When interpreting such commands, the underlying text appears as a sequence of tokens like name,
number, special symbol etc. much rather than as a sequence of characters.  Therefore,
we have adopted the well-known concepts of syntax and scanning from the discipline
of compiler construction, including functional support.
The Oberon scanner recognizes tokens of some universal classes.
They are name, string, integer, real, longreal, and special character.
The exact syntax of universal Oberon tokens is:
\begin{verbatim}
  token = name | string | integer | real | spexchar
  name  = ident {.ident}
  ident = letter { letter | digit }
  string = '"' { char } '"'
  integer = [+|-] number
  real = [+|-] number . number [E [+|-] number]
  number = digit { digit }
  spexchar = any character except letters, digits,
                   space, tab, or carriage-return
\end{verbatim}
\verb|Scanner| is defined correspondingly as
\begin{verbatim}
  Scanner = RECORD (Reader) nextCh: CHAR;
              line, class, i, len: INT;
              x: REAL; c: CHAR;
              s: ARRAY 32 OF CHAR
            END;
\end{verbatim}
This type is actually a variant record type with class as discriminating tag.
Depending on its class the value of the current token is stored in one of the fields
\verb|i|, \verb|x|, \verb|c|, or \verb|s|.
\begin{itemize}
  \item \verb|len| gives the length of \verb|s|,
  \item \verb|nextCh| typically exposes the character terminating the current token, and
  \item \verb|line| counts the number of lines scanned.
\end{itemize}

The operations on scanners:
\begin{verbatim}
PROC OpenScanner(VAR S: Scanner; T: Text; pos: INT);
PROC Scan       (VAR S: Scanner);
\end{verbatim}

They correspond exactly to their counterparts \verb|OpenReader| and \verb|Read| respectively.
Writers are dual to readers.  They serve the purpose of creating and extending texts.
However, again, they do not operate on texts directly.  Rather,
they act as self-contained aggregates, continuously consuming and buffering textual data.

The formal declaration of \verb|Writer| resembles \verb|Reader|:
\begin{verbatim}
  Writer = RECORD buf: Buffer;
             fnt: Fonts.Font; col, voff: INT
           END;
\end{verbatim}

\verb|buf| is an internal buffer containing the consumed data. \\
\verb|fnt|, \verb|col|, and \verb|voff| specify the current looks
for the next character consumed by this writer.

The following procedures constitute the \verb|Writer| API:
\begin{verbatim}
  PROC OpenWriter(VAR W: Writer);
  PROC SetFont   (VAR W: Writer;  fnt: Fonts.Font);
  PROC SetColor  (VAR W: Writer;  col: INT);
  PROC SetOffset (VAR W: Writer; voff: INT);
\end{verbatim}
\verb|OpenWriter| opens a new writer with an empty buffer. \\
\verb|SetFont|, \verb|SetColor|, and \verb|SetOffset| set the respective current look.
For example, \verb|SetFont(W, fnt)| is equivalent with \verb|W.fnt := fnt|.
These procedures are included because
\verb|fnt|, \verb|col|, and \verb|voff| are read-only for clients.

The question may arise how data is produced and transferred to writers.
The answer is a set of writer procedures, each of them handling an individual data type:
\begin{verbatim}
PROC Write       (VAR W: Writer; ch: CHAR);
PROC WriteLn     (VAR W: Writer);
PROC WriteString (VAR W: Writer; s: ARRAY OF CHAR);
PROC WriteInt    (VAR W: Writer; x, n: INT);
PROC WriteHex    (VAR W: Writer; x: INT);
PROC WriteReal   (VAR W: Writer; x: REAL; n: INT);
PROC WriteRealFix(VAR W: Writer; x: REAL; n, k: INT);
PROC WriteClock  (VAR W: Writer; d: INT);
\end{verbatim}

The following is schematic fragment of a client program that creates textual output:
\begin{verbatim}
  open writer; set desired font;
  REPEAT
    process;
    write result to writer;
    append writer buffer to output text
  UNTIL ended
\end{verbatim}
Of course, writers can be reused.
For example, a single global writer is typically shared by all of the procedures within a module.
In this case, the writer needs to be opened just once at module loading time.

Typically, however, accessing aggregates are of a transient nature and are bound to a certain activity,
which manifests itself in their allocation on the stack
without any possibility of referencing them from the outside of the activity,
in contrast to the underlying texts that are allocated on the system heap and have a much longer life time.

Let us summarize:
\begin{itemize}
  \item \verb|Text| in Oberon is a powerful abstract data type with intrinsic operations from 3 areas:
    \begin{enumerate}
      \item Loading/storing,
      \item editing, and
      \item accessing (reading/writing).
    \end{enumerate}
  \item The latter 2 areas on their part introduce further abstract types
    \verb|Buffer|, \verb|Reader|, \verb|Scanner|, and \verb|Writer|.

    In combination they guarantee a clean separation of very different concerns.
    The benefits of such a rigorous decoupling are numerous.
    For example, it makes it possible to freely choose (and vary) the granularity
    at which a text and its views are updated.
  \item Finally, an after-method is used to allow context-dependent post-processing of editing operations.
    It is used primarily for preserving consistency between text models and their views.
\end{itemize}
