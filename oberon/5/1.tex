\section{Text as an abstract data type}
\label{sec:textyp}
The concept of abstraction is arguably the most important achievement of PL development.
It provides a powerful tool to create simplified views of complicated things and
connections. Two prominent examples of program abstractions are definitions (interfaces) and
abstract data types, embodying simplified views on a certain piece of program and on a certain kind
of data respectively.

We shall now give a precise definition of the notion of text in Oberon by presenting it as an abstract
data type. It is important not to confuse this type with the far less powerful type String as it is often
supported by advanced PLs. In this Section we carefully avoid revealing any
implementation aspects of the abstract type Text. Our viewpoint is that of an application program
operating on text abstractly or using it as a medium of communication.

Nevertheless, let us first use a symbolic looking glass to get a refined understanding of the concept
of character in the context of rich texts. We know that each character represents a textual element
of information. If displayed, it also refers to some specific graphical pattern, often called glyph. In
Oberon, we do justice to both aspects by thinking of the ASCII-code as an index into a font that is
into a set of glyphs of the same style. Representing characters as pairs (font, ref), where font
designates a font and ref the character's ASCII-code and adding two more attributes color and
vertical offset, we get to a quadruple representation (font, ref, col, voff) of characters. The
components font, color, and vertical offset together are often referred to as looks. With that, we can
now define a (rich) text as a sequence of characters with looks. We shall treat the topic of fonts and
glyphs thoroughly in Section 5.4.

For the moment, however, let us continue our discussion of the abstract data type Text. Formally,
we define it as
\begin{verbatim}
  Text = POINTER TO TextDesc;
  TextDesc = RECORD
    len: INT;
    notify: Notifier
  END;
\end{verbatim}
There is only one state variable and one method. The variable len represents the current length of
the described text (i.e. the number of characters in the sequence). The procedure variable notify is
included as a method (occasionally called after-method) to notify interested clients of state
changes.

By definition, each abstract data type comes with a complete set of operations. In the case of Text,
three different groups corresponding to three different topics need to be considered, loading (from
file), storing (to file), editing, and accessing (reading and writing) respectively.

\subsection{Loading and Storing Text}
Let us start with the file group. We first introduce a pair of mutually inverse operations called
internalize and externalize. Their meaning is "load from file and build up an internal data structure"
and "serialize the internal data structure and store it on file" respectively. There are three
corresponding procedures:
\begin{verbatim}
  PROC Open (T: Text; name: ARRAY OF CHAR);
  PROC Load (T: Text; f: Files.File;
                  pos: INT; VAR len: INT);
  PROC Store (T: Text; f: Files.File;
                  pos: INT; VAR len: INT);
\end{verbatim}
Logical entities like texts are stored in Oberon on external media in the form of sections. A section
is addressed by a pair (file, pos) consisting of a file descriptor and a starting position. In general, the
structure of sections obeys the following syntax:
\begin{verbatim}
  section = identification type length contents.
\end{verbatim}

Procedure Open internalizes a named text file (consisting of a single text section), procedure Load
internalizes an arbitrary text section starting at (f, pos), and procedure Store externalizes a text
section to (f, pos). The parameter T designates the internalized text. len returns the length of the
section. Note that in case of Load the identification of the section must have been read and
consumed before the loader is called.

\subsection{Editing Text}
Our next group of operations supports text editing. It comprises four procedures:
\begin{verbatim}
  PROC Delete (T: Text; beg, end: INT);
  PROC Insert (T: Text; pos:INT; B:Buffer);
  PROC Append (T: Text; B: Buffer);
  PROC ChangeLooks (T: Text; beg, end: INT;
     sel: SET; fnt: Fonts.Font; col, voff: INT);
\end{verbatim}
Again, we should first explain the types of parameters. Procedures Delete and ChangeLooks each
take a stretch of text as an argument which, by definition, is an interval [beg, end) within the given
text. In the parameter lists of Insert and Append we recognize a new data type Buffer.
Buffers are a facility to hold anonymous sequences of characters. Type Buffer presents itself again
as an abstract data type:
\begin{verbatim}
  Buffer = POINTER TO BufDesc;
  BufDesc = RECORD len: INT END;
\end{verbatim}
\verb|len| specifies the current length of the buffered sequence. The following procedures represent the intrinsic operations on buffers:
\begin{verbatim}
  PROC OpenBuf (B: Buffer);
  PROC Copy (SB, DB: Buffer);
  PROC Save (T:Text; beg,end:INT; B:Buffer);
\end{verbatim}

Their function is in turn opening a given buffer B, copying a buffer SB to DB, saving a stretch [beg,
end) of text in a given buffer, and recalling the most recently deleted stretch of text and putting it
into buffer B.

Buffer is used as an auxiliary data type in editing procedures. Procedure Delete deletes the given
stretch [beg, end) within text T, Insert inserts the buffer's contents at position pos within text T, and
Append(T, B) is a shorthand form for Insert(T, T.len, B). Note that, as a side-effect of Insert and
Append, the buffer involved is emptied. Finally, procedure ChangeLooks allows to change selected
looks within the given stretch [beg, end) of text T. sel is a mask selecting a subset of the set of
looks { font, color, vertical offset }.

It is time now to come back to the notifier concept. Recapitulate that notify is an “after-method”. It
must be installed by the client when opening the text and is called at the end of every editing
operation. Its signature is
\begin{verbatim}
  Notifier = PROC(T:Text; op,beg,end:INT);
\end{verbatim}
The parameters \verb|op|, \verb|beg|, and \verb|end| report about the operation (\verb|op|) that calls the notifier and on the
affected stretch \verb|[beg, end)| of the text. There are three different possible variants of \verb|op|
corresponding to the three different editing operations: \verb|op =| delete, insert, replace correspond to
procedures \verb|Delete|, \verb|Insert|(, \verb|Append|), and \verb|ChangeLooks| respectively.

By far the most important application of the notifier is updating the display, i.e. adjusting all affected
views of the text that are currently displayed to the new state of the text (the model). We shall come
back to this important matter when discussing text frames in Section 5.3.

In concluding this Section it is worth noting that the groups of operations just discussed have been
designed to be equally useful for interactive text editors as for programmed text generators/manipulators.

\subsection{Accessing Text}
Let us now turn to the third and last group of operations on texts: Accessing that is reading and
writing. According to the principle of separation of concerns, one of our guiding principles, the
access mechanism operates on extra aggregates called readers and writers rather than on texts
themselves.

Readers are used to read texts sequentially. Their type is declared as
\begin{verbatim}
  Reader = RECORD
    eot: BOOL; (*end of text*)
    fnt: Fonts.Font;
    col, voff: INT
  END;
\end{verbatim}

A reader must first be opened at the desired position in the text before it can then be moved
forward incrementally by reading character-by-character. Its state variables indicate end-of-text and
expose the looks of the character last read.
The corresponding operators are
\begin{verbatim}
  PROC OpenReader (VAR R: Reader; T: Text;
                                     pos: INT);
  PROC Read (VAR R: Reader;  VAR ch: CHAR);
\end{verbatim}

Procedure \verb|OpenReader| sets up a reader \verb|R| at position \verb|pos| in text \verb|T|. Procedure \verb|Read| returns the
character at the current position of \verb|R| and makes \verb|R| move to the next position.
The current position of reader \verb|R| is returned by a call to the function \verb|Pos|:
\begin{verbatim}
  PROC Pos (VAR R: Reader): INT;
\end{verbatim}

In Chapter 3 we learned that commands plus parameter lists are often embedded in ordinary texts.
When interpreting such commands, the underlying text appears as a sequence of tokens like name,
number, special symbol etc. much rather than as a sequence of characters. Therefore, we have
adopted the well-known concepts of syntax and scanning from the discipline of compiler
construction, including functional support. The Oberon scanner recognizes tokens of some
universal classes. They are name, string, integer, real, longreal, and special character.
The exact syntax of universal Oberon tokens is:
\begin{verbatim}
token = name | string | integer | real | spexchar.
name = ident { "." ident }.ident
     = letter { letter | digit }.
string = """ { char } """.
integer = ["+"|"-"] number.
real = ["+"|"-"] number "." number ["E" ["+"|"-"] number].
number = digit { digit }.
spexchar = any character except letters, digits,
                space, tab, and carriage-return.
\end{verbatim}
Type Scanner is defined correspondingly as
\begin{verbatim}
  Scanner = RECORD (Reader)
    nextCh: CHAR;
    line: INT;
    class: INT;
    i: INT;
    x: REAL;
    c: CHAR;
    len: INT;
    s: ARRAY 32 OF CHAR
  END;
\end{verbatim}

This type is actually a variant record type with class as discriminating tag. Depending on its class
the value of the current token is stored in one of the fields i, x, c, or s. len gives the length of s,
nextCh typically exposes the character terminating the current token, and line counts the number of
lines scanned.

The operations on scanners are
\begin{verbatim}
  PROC OpenScanner (VAR S: Scanner; T: Text;
                                     pos: INT);
  PROC Scan (VAR S: Scanner);
\end{verbatim}

They correspond exactly to their counterparts OpenReader and Read respectively.
Writers are dual to readers. They serve the purpose of creating and extending texts. However,
again, they do not operate on texts directly. Rather, they act as self-contained aggregates,
continuously consuming and buffering textual data.

The formal declaration of type Writer resembles that of type Reader:
\begin{verbatim}
  Writer = RECORD
    buf: Buffer;
    fnt: Fonts.Font;
    col, voff: INT
  END;
\end{verbatim}

\verb|buf| is an internal buffer containing the consumed data. \verb|fnt|, \verb|col|, and \verb|voff| specify the current looks for
the next character consumed by this writer.

The following procedures constitute the Writer API:
\begin{verbatim}
  PROC OpenWriter (VAR W: Writer);
  PROC SetFont (VAR W: Writer; fnt:Fonts.Font);
  PROC SetColor (VAR W: Writer; col: INT);
  PROC SetOffset (VAR W: Writer; voff:INT);
\end{verbatim}

Procedure OpenWriter opens a new writer with an empty buffer. Procedures SetFont, SetColor, and
SetOffset set the respective current look. For example, SetFont(W, fnt) is equivalent with W.fnt :=
fnt. These procedures are included because fnt, col, and voff are read-only for clients.

The question may arise how data is produced and transferred to writers. The answer is a set of
writer procedures, each of them handling an individual data type:
\begin{verbatim}
PROC Write (VAR W: Writer; ch: CHAR);
PROC WriteLn (VAR W: Writer);
PROC WriteString (VAR W:Writer; s:ARRAY OF CHAR);
PROC WriteInt (VAR W: Writer; x, n: INT);
PROC WriteHex (VAR W: Writer; x: INT);
PROC WriteReal (VAR W:Writer; x:REAL; n:INT);
PROC WriteRealFix (VAR W: Writer; x: REAL;
                                    n, k: INT);
PROC WriteClock(VAR W: Writer; d: INT);
\end{verbatim}

The following is schematic fragment of a client program that creates textual output:
\begin{verbatim}
  open writer; set desired font;
  REPEAT
    process;
    write result to writer;
    append writer buffer to output text
  UNTIL ended
\end{verbatim}
Of course, writers can be reused. For example, a single global writer is typically shared by all of the
procedures within a module. In this case, the writer needs to be opened just once at module loading
time.

Typically, however, accessing aggregates are of a transient nature and are bound to a certain
activity, which manifests itself in their allocation on the stack without any possibility of referencing
them from the outside of the activity, in contrast to the underlying texts that are allocated on the
system heap and have a much longer life time.

Let us summarize: Text in Oberon is a powerful abstract data type with intrinsic operations from
three areas: Loading/storing, editing, and accessing (reading/writing). The latter two areas on their
part introduce further abstract types called Buffer, Reader, Scanner, and Writer. In combination
they guarantee a clean separation of very different concerns. The benefits of such a rigorous
decoupling are numerous. For example, it makes it possible to freely choose (and vary) the
granularity at which a text and its views are updated. Finally, an after-method is used to allow
context-dependent post-processing of editing operations. It is used primarily for preserving
consistency between text models and their views.
