\section{The Edit toolbox}
We have seen that every text frame integrates an interactive text editor that we can regard as an
interpreter of a set of built-in commands (intrinsic commands). Of course, we would like to be able
to extend this set by custom editing commands (extrinsic commands). Adding additional editing
commands was indeed a worthwhile stress test for the underlying texts API. Module \verb|Edit| is the
result of this effort. It is a toolbox of consisting of some standard extrinsic editing commands.
\begin{verbatim}
  DEFINITION Edit;
    PROC Open; (*text viewer*)
    PROC Show; (*text*)
    PROC Locate; (*position*)
    PROC Search; (*pattern*)
    PROC Store; (*text*)
    PROC Recall; (*deleted text*)
    PROC CopyFont;
    PROC ChangeFont;
    PROC ChangeColor;
    PROC ChangeOffset;
  END Edit.
\end{verbatim}

The first group of commands in \verb|Edit| is used to display, locate, and store texts or parts
of texts. In turn they open a text file and display it, open a program text and show the declaration
of a given object, locate a given position in a displayed text (main application: locating an error
found by the compiler), search a pattern, and store the current state of a displayed text. Commands
in the next group are related with editing. They allow restoring of the previously deleted part of text,
copying a font attribute to the current text selection, and change attributes of the current text
selection. Note that the commands \verb|CopyFont|, \verb|ChangeFont|, \verb|ChangeColor|, and
\verb|ChangeOffset| are extrinsic variations of the intrinsic copy-look operation. The implementations
of the toolbox commands are given in the Appendix.

\section*{References}
\begin{description}
  \item[Gutknecht] J. Gutknecht, "Concept of the Text Editor Lara", Communications of the ACM,
                   Sept. 1985, Vol.28, No.9.
  \item[Teitelman] W. Teitelman, "A tour through Cedar", IEEE Software, 1, (2), 44-73 (1984).
\end{description}
