\setcounter{section}{-1}
\section{Introduction}
\label{ch:intro}
The idea for this project has two roots. The 1st was a project to design and implement a small
processor for use in embedded systems with several interconnected cores. It was called the Tiny
Register Machine (TRM). The 2nd root is a book written by me some 30 years ago, and revised
several times since. Its subject is Compiler Construction. The target for the developed
compiler is a hypothetical computer. In the early years this computer was a stack architecture,
later replaced by a RISC (Reduced Instruction Set Computer) architecture. Now the intent is to
replace the hypothetical, emulated computer by a real one. This idea was made realistic by the
advent of programmable hardware components called Field Programmable Gate Arrays (FPGA).

The development of this architecture is described in detail in this report. It is intentionally held in a
tutorial style and should provide the reader with an insight into the basic concepts of processor
design. In particular, it should connect the subjects of architectural and compiler design, which are
so closely interconnected.

We are aware of the fact that “real” processors are far more complex than the one presented here.
We concentrate on the fundamental concepts rather than on their elaboration. We strive for a fair
degree of completeness of facilities, but refrain from their “optimization”. In fact, the dominant part of
the vast size and complexity of modern processors and software is due to speed-up called
optimization (improvement would be a more honest word). It is the main culprit in obfuscating the
basic principles, making them hard, if not impossible to study. In this light, the choice of a RISC
 is obvious.

The use of an FPGA provides a substantial amount of freedom for design. Yet, the hardware
designer must be much more aware of availability of resources and of limitations than the software
developer. Also, timing is a concern that usually does not occur in software, but pops up
unavoidably in circuit design. Nowadays circuits are no longer described in terms of elaborate
diagrams, but rather as a formal text. This lets circuit and program design appear quite similar. The
circuit description language — we here use Verilog — appears almost the same as a programming
language. But one must be aware that differences still exist, the main one being that in software we
create mostly sequential processes, whereas in hardware everything “runs” concurrently. However,
the presence of a language — a textual definition — is an enormous advantage over graphical
schemata. Even more so are systems (tools) that compile such texts into circuits, taking over the
arduous task of placing components and connecting them (routing). This holds in particular for
FPGAs, where components and wires connecting them are limited, and routing is a very difficult and
time-consuming matter.

The development of our RISC progresses through several stages. The 1st is the design of the
architecture itself, (more or less) independent of subsequent implementation considerations. Then
follows a 1st implementation called RISC-0. For this we chose a Harvard Architecture, implying that
2 distinct memories are used for program and for data. For both we use chip-internal memory, so-called
Block RAMs (BRAM). The Harvard architecture allows for a neat separation of the arithmetic from the
control unit.

But these BRAMs are relatively small (1 - 4K words). The development board used in this project,
however, provides an FPGA-external Static RAM (SRAM) with a capacity of 1 MByte. In the 2nd stage
(RISC-1) the BRAM for data is replaced by this SRAM.

In the 3rd stage (RISC-2) we switch to a von Neumann architecture. It uses a single memory for
both instructions and data. Of course we use the SRAM of stage 1.

In the 4th stage (RISC-3) 2 features are added to round up the development, byte accessing and
interrupts. (They might be omitted in a course without consequences).

RISC-4 adds non-volatile storage (disk) in the form of an SD-card (flash memory). This device is
accessed through a serial line according to the SPI (Serial Peripheral Interface) standard.

RISC-5 finally adds further peripheral devices. They are a keyboard and a mouse (both accessed
via standard PS-2 interface), a memory-mapped display (accessed via VGA standard), and a network
(accessed via SPI standard).

It follows that RISC-0 and RISC-5 are the corner stones of this project. But we keep in mind that
from a programmers’ standpoint all implementations appear identical (with the exception of the
features added in RISC-3). All 4 have the same architecture and the same instruction set.
