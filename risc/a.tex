\section{Revised RISC0\&5}
v5.12.10

\subsection{Resources and registers}
From the viewpoints of the programmer and the compiler designer the computer consists of an Arithmetic Unit (AU),
a Control Unit (CU) and a store (memory).
\begin{itemize}
  \item The AU contains 16 registers R0-R15, with 32 bits each.
  \item The CU consists of the Instruction Register (IR), holding the instruction currently being executed, and
    the Program Counter (PC), holding the word-address of the instruction to be fetched nextly. All Branch
    Instructions (BI) are conditional.
  \item The memory consists of 32-bit words, and it is byte-addressed. Furthermore, there are 4 flag registers N,
    Z, C and V, called the Condition Codes (CC).
\end{itemize}

There are 3 types of instructions and 4 instruction formats (F0-F3):
\begin{itemize}
  \item Register Instructions (RI) operate on registers only and feed data through a shifter or the
    Arithmetic Logic Unit (ALU) (formats F0 and F1).
  \item Memory instructions (MI) fetch and store data in memory (format F2).
  \item BIs affect the PC (format F3).
\end{itemize}

\subsection{RI}
RIs assign the result of an operation to the destination register $R.a$. The 1st operand is the register $R.b$
and the 2nd operand $n$ is either register $R.c$ (format F0) or a constant $im$ (format F1).

Immediate values are extended to 32 bits with 16 v-bits to the left. Apart from $R.a$, these instructions also
affect the flag registers N(egative) and Z(ero). The ADD and SUB instructions also set the flags C(arry, borrow)
and V (oVerflow).

\subsection{MI}
If $v = 0$, access is for a word (4 bytes). If $v = 1$, a single byte is accessed.

\subsection{BI}
If $u = 0$, the destination address is taken from register $R.c$. If $u = 1$, it is $PC + 1 + offset$. If $v = 1$,
the link address $PC + 1$ is deposited in the register $R15$.

\subsection{Special features}
Modifier bit $u = 1$ changes the effect of certain instructions as follows:
\begin{verbatim}
  ADD', SUB' add, subtract also carry C
  MUL' unsigned multiplication
  MOV' form 0, v = 0: R.a := H
  MOV' form 0, v = 1: R.a := [N, Z, C, V]
  MOV' form 1 R.a := [im 16'b0] (im left shifted 16 bits)
\end{verbatim}

The MUL instruction deposits the high 32 bits of the product in the auxiliary register H. The DIV instruction
deposits the remainder in H.

\subsection{Interrupts}
The addition of an interrupt facility required the addition of 2 new instructions, as well as the status register
$intenb$ (interrupt enable). The instructions are
\begin{verbatim}
  RTI     1100 0111 xxxx xxxx xxxx xxxx 0001  Rn
                         // return from interrupt
  STI/CLI 1100 1111 xxxx xxxx xxxx xxxx 0010 000e
              // set/clear interrupt, intenb := e
\end{verbatim}

These instructions are encoded as BIs with bits 4 or 5 set.

\subsection{RISC0}
The RISC architecture has been implemented on a Xilinx FPGA contained on the development board Spartan. RISC0
stands at the origin of an evolving series of extensions. It represents a Harvard architecture, and it uses
FPGA-internal RAM for its memory, which is restricted to 8K words of program and 8K words for data. It does not
feature byte access, and the floating-point instructions are not available.

RISC0's external devices are the following:
\begin{table}[h!]
  \centering
  \begin{tabular}{r c l l}
    adr &   hex   & input               & output \\\hline
    -64 & 0FFFC0H & millisecond counter & \\
    -60 & 0FFFC4H & switches            & LEDs \\
    -56 & 0FFFC8H & RS-232 data         & RS-232 data \\
    -52 & 0FFFCCH & RS-232 status\footnote{}
                                        & RS-232 control
  \end{tabular}
\end{table}

{\footnotesize 1 bit $0$: receiver ready,
                 bit $1$: transmitter ready.}

\subsection{RISC5}
RISC5 is an extension of RISC0 based on a von Neumann architecture and uses the same instruction set. The memory
consists of the board-internal SRAM with a capacity of 1 MB. Byte access is available, and so are floating-point
instructions.

RISC5's external devices are the following:
\begin{table}[h!]
  \centering
  \begin{tabular}{r c l l}
    adr &   hex   & input                & output \\\hline
    -64 & 0FFFC0H & millisecond counter  \\
    -60 & 0FFFC4H & switches             & LEDs \\
    -56 & 0FFFC8H & RS-232 data          & RS-232 data \\
    -52 & 0FFFCCH & RS-232 status        & RS-232 control \\
    -48 & 0FFFD0H & disk, net SPI data   & SPI data \\
    -44 & 0FFFD4H & disk, net SPI status & SPI control \\
    -40 & 0FFFD8H & keyboard data (PS2) \\
    -36 & 0FFFDCH & mouse (and kbd status)
  \end{tabular}
\end{table}

A further added device is the video controller. It maps memory at 0E7F00H - 0FFEFFH onto the display (1024 x 768
pixels).
