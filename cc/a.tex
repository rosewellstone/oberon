\section{1975: Micro Computers}
Microcomputers mentioned in the preceding paragraph were the visible result of
the immense progress made in semiconductor technology, in particular of the
efforts on miniaturization. Discrete transistors had been replaced by integrated
circuits containing themselves many transistors, all produced in the same
manufacturing step in a silicon foundry. Fairchild was leading the way, followed by
Intel, National Semiconductor, Motorola, and others. First, there was the (short-
lived) RTL technology (Resistor-Transistor Logic), soon to be super ceded by TTL
technology (Transistor-Transistor Logic). Packages of ICs successfully became
standardized founding the era of TTL chips, the series 74xxx. (Only the military
had their own, more expensive cookies, the 54xxx series). Also, a single supply
voltage of 5V belonged to the standard (only at the beginning augmented by -5V
and 12V for memory chips). This was definitely one of the most successful
standardization efforts in industrial history.

The building blocks of circuits were no longer transistors and resistors (and
occasionally a capacitor), but elementary circuits, such as gates, multiplexers,
decoders, adders, register arrays, and buffers. The heart of these micro computers
was a single chip of a novel dimension of complexity, a complete, small computer,
incorporating a simple arithmetic/logical unit, a set of data registers, an instruction
register and a program counter, that is, a complete control unit.

The (not quite) first microprocessor chips featured a data path of only 8 bits. The
prominent samples were the Intel 8080, the Motorola 6800, and the Rockwell
6502. Memory chips became available, first with 1K bits, then followed soon by 4K,
16K and even 64K (1980). As a result of this development it became relatively
easy to design and build small computers for modest amounts of money. Upgrades
of microprocessors followed soon, in particular the Motorola 6809 with a 16-bit
internal ALU.

A special line of microcomputers appeared soon thereafter (1975). They were
complete computer systems on a single chip, and they became known as micro-
controllers, to be used mostly in embedded systems. They consisted of a simple
ALU, a control unit, and a small amount of static memory (SRAM). They also
contained on-chip (programmable) program memory. In early versions, this
memory was writable only once (PROM), later versions contained erasable
memory (EPROM). The most successful ones came from Intel (8048, 8051) and
they were soon manufactured by the millions, driving down the cost to the order of
a dollar and entering cars, refridgerators, and television sets.
