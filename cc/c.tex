\section{1975: Beginning of Computing-Age}
Micro computers quickly spread into homes and schools and made computing a "household
activity”. Still, they had to be considered as toys. They were insufficiently powerful
for the task of serious work. The 1st truly useful Personal Computer (PC) I
encountered during a sabbatical year at the Palo Alto Research Center (PARC) of Xerox,
was the \emph{Alto} (B. Lampson, Ch. Thacker, E. McCreight).

The Alto did not use one of these 8-bit microprocessors, but was built with discrete
components of the 74Sxx series of chips. The ALU's heart were 4 ALU chips 748181 with
a propagation delay of less than 100 ns. The data paths were 16 bits wide, and the
memory had a capacity of 64K 16-bit words. The Alto was equipped with a 2 MB cartridge
disk store. It featured a bit-mapped display, 808 bits high and 606 dots (bits) wide,
and a novel pointing device called mouse with 3 buttons. This latter combination
allowed to directly program every dot on the entire display field. This was in stark
and visible contrast to the conventional displays with fixed character sets, 80
characters per line and 25 lines on the page. Texts could be displayed with
individually designed character patterns, the basis for the use of many fonts in
regular, italic, and bold styles. Arbitrary graphics could be displayed mixing texts,
line graphics, and small pictures. These features opened an entirely new world for
programmers and computers. The fundamental innovation was interactivity.

Equally important was the fact that the memory size was sufficiently large to
accommodate compilers for high-level languages for system programming. At Xerox, the
language Mesa covered these needs. It represented a big extension of Pascal with
several features deemed necessary to express particular facilities of the Alto
hardware. Mesa definitely belonged to the 2nd, if not 3rd generation of programming
languages. Like the Alto and its mouse, it was Xerox-proprietary.

The implementation of Mesa rested on Mesa byte code. The compiler generated byte code
(similar to Pascal's earlier P-code) and the Alto was micro-programmed to interpret
its byte code. This interpreter resided permanently in a special, very fast microcode
memory. This scheme allowed for a rather dense program code, and only through this
design large programs could be fitted into the still rather limited main memory.

Another hallmark of the Alto was that all workstations were connected through a
network, the 3-MHz Ethernet, a single wire bus. The concept of servers emerged. There
was a server, a dedicated Alto, for a 1st laser printer, and one for a large, common
file store.

What so far had been possible only on large-scale mainframes, was now feasible on a
personal, that is, not shared computer. The new scheme was such a radical departure
from the common computing environment that it is fair to call it the beginning of the
(modern) computing era. After all, every one of the millions of computer users now
use PCs (laptops) whose ancestor is the Alto. Hardly anybody remembers or imagines
the way computers were used before 1975, namely through card decks or slow lines
connected to terminals. The difference between then and now is enormous.
