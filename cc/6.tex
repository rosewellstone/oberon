\section{1963: B-5000}
\label{sec:B5K}
It was to be hoped that the innovative design of Algol would extend its influence on
computer development. And it did. Algol turned out to be difficult, if not impossible,
to implement efficiently and to make it competitive with Fortran concerning
execution speed. Already in the early 1960s some computers appeared that
catered to Algol's specific features, emphasis given to the implementation of
general expressions and of recursive procedures.

In order to translate expressions of Algol's generality into sequences of simple
instructions, they are typically converted from infix to postfix form:
\begin{table}[h!]
  \centering
  \begin{tabular}{l l}\\
    infix         & postfix \\\hline
    x + y         & x y + \\
    x + y + z     & x y + z + \\
    (x+y) * (z-w) & x y + z w - *
  \end{tabular}
\end{table}

This allows a straight-forward evaluation, if operands can be stacked (pushed onto
a first-in-last-out stack). Then operators simply replace operands by the result.
Computer designers therefore accommodated compiler designers by providing a
push-down stack in place of a register array. Effectively, the stack is an array with
implied up-down counter used as index, and the net effect is that register numbers
can be omitted in the code. This leads to denser code.
The idea of a (hardware) stack appeared as very attractive, and both the British
and the Dutch Electronica X8 implemented it. The Burroughs B5000 also
adopted the concept, implementing the top two elements as registers. Transfer of
lower elements of the stack to and from memory was automated by a fairly
complex scheme.

It is noteworthy that the B5000 used the same encoding for integers and real
numbers with a 39-bit mantissa, a 6-bit exponent, and a base B = 8, integers with
exponent = 0. The program code consists of 12-bit syllables, 4 to a word.

In the long run, the idea of the expression stack did not catch on and disappeared
form the scenery. Direct register numbers in the code seemed preferable, and
sophisticated optimization algorithms make the best of it.

The most challenging feature of Algol, however, were recursive calls of
procedures. Consider
\begin{verbatim}[language=Algol]
  PROCEDURE P(x: INTEGER);
  BEGIN
    IF x > 0 THEN P(x-1) END
  END P
\end{verbatim}

For every (recursive) call of P a new incarnation of $x$ is created and must be
allocated. The same would hold for local variables. The evident solution is a stack
of frames, each holding parameters, local variables, and the return address. As a
consequence, every variable cannot be accessed by a simple, static address, but
by an address, to which the base address, the address of the containing frame, is
added. This base must be available from an index register, to be initialized upon
call of the procedure (and restored on exit).

A nasty problem with existing main frame computers was the acquisition of the
return address (also to be deposited in the frame). The worst solution was that of
the CDC 6000 super computer. Its call instruction deposited the return address in
the memory location preceding the call address (i.e. in front of the procedure's first
instruction. This scheme writes into the program, and thereby makes the code
non-reentrant. Clearly an unacceptable solution.

The efficient and elegant implementation of recursion requires a suitable
addressing mode with specific address registers (stack pointer, frame pointer). The
B5000 offers these ingredients. As a result, however, a call instruction constitutes
a fairly complicated scheme with several accesses to memory, requiring complex
circuitry and time.

But Algol had some further hard nuts in store. A major stumbling block was its
so-called name-parameter. It postulated the literal replacement of the formal
parameter by the actual parameter. The best example for its use is the following
procedure (here translated into the syntax of Pascal):
\begin{verbatim}[language=Pascal]
  PROCEDURE Sum(i,n: INTEGER; x: REAL): REAL;
    VAR s: REAL:
  BEGIN s := 0;
    FOR i := 0 TO n-1 DO s := x+s END;
    RETURN s
  END Sum;
\end{verbatim}
Given further variables
\begin{verbatim}[language=Pascal]
  k: INTEGER;
  a: ARRAY 100 OF REAL;
\end{verbatim}
the sum of the elements of $a$ can be expressed by the function call
\begin{verbatim}[language=Pascal]
  Sum(k, 100, a[k])
\end{verbatim}
rendering the FOR statement into
\begin{verbatim}[language=Pascal]
  FOR k := 0 TO 100-1 DO s := a[k] + s END
\end{verbatim}

This implies that for every addition the actual parameter $a[k]$ must be evaluated,
and that therefore it must be called like a (parameter-less) function. The feature
complicates compilation of procedure calls enormously. The B-5000 pushed this
complication from the compiler (software) into hardware: Instructions for fetching a
value must first inspect (at run-time) whether an operand is a variable or a function
call. Discrimination was based on a bit in every operand, indicating whether it is a
simple variable or a descriptor (of the function).

The B5000 introduced the concept of descriptors and is said to have a Descriptor
architecture. In particular, every array is represented by an array descriptor, just as
procedures are represented by procedure descriptors. The array descriptor also
contains the array index bounds, and an array access tests the index to be within
these bounds. The net effect of the descriptor scheme was that almost every
variable access required an indirection, and therefore more time.

In most cases, computer engineers had little notion about languages and
compilers, and language designers had little knowledge about hardware design
and even less about compiling. They were even proud of this and believed that
true innovation could occur only by ignoring technical constraints. The result was a
gap between software and hardware. The B5000 designers tried to bridge this gap
courageously. But in hindsight they went too far. They implemented language
features that proved to be not quite realistic, nor truly needed. As a consequence,
the B5000 computer, delivered in 1964, one of the first to Stanford University,
became a very complex, costly device. It could efficiently handle sophisticated
Algol features, but it could hardly compete with conventional machines for
everyday applications.
