\section{1964: Family Notion and 8-bit Byte}
IBM's System 360 was announced in 1964. It brought 2 other innovations. The
1st was the notion of a computer family. Up to this time, every computer had its
own structure and performance; it was sort of unique. Now System 360 consisted
of many incarnations of one and the same instruction set. Each incarnation, called
model, had its distinct performance figures, size and price. But all of them looked
the same to the programmer. In this sense they formed a family; they featured the
same architecture. This is where the word architecture in connection with
computers appeared.

It goes without saying that the different models had very different implementations
of the hardware, although they were handling the same software. For the first time,
the technique of emulation was used extensively. The smaller models used the
novel technique of micro-programs. The genuine hardware always interpreted the
same program, namely an interpreter of the 360 instruction set. The interpreter
was defined in so-called microcode. This micro-code rested in a small but very fast
microcode memory, typically a read-only memory implemented in a proprietary
technology. The consequence was that some of the 360's instructions were fast,
others slow in comparison. It was possible to include some very complex and hard
to understand instructions. Some of them even included loops in the micro-code.

The 2nd novelty was that the smallest individually addressable unit in memory
was not the word, but the byte, and that this byte consisted of 8 bits. So far, that
unit was considered as consisting of 6 bits, and the word length of all computers
were multiples of 6. An immediate negative consequence had been the limitation of
character sets (ASCII and IBM's EBCDIC) to 64 characters. The extension to 256
characters was a welcome benefit.

This revolutionary step also dispensed with "variable length data computers", such
as the ubiquitous IBM 1401, which featured string instructions, which operated on
sequences of 9-bit bytes, the 9th bit acting as a string termination indicator.
