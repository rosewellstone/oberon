\section{Correctness Concerns}
The tremendous increase of computing power clearly resulted in rapidly growing
demands on software which became larger and more intricate, but still being
produced by old methods and teams. Gradually it became admitted that programming
was a difficult intellectual activity, and that sometime had to be done to
curtail difficulties, that a discipline was required.

The leading voice in this new insight was E. W. Dijkstra, well-known for his
slashing of the \emph{goto} statement, leading to undisciplined program structures.
His 1st contribution was to establish a style of \emph{structured programming},
implying less \emph{goto} programs. But he also tackled and categorized problems
arising with the new discipline of \emph{concurrent programming}, where several
programs run at the same time, communicating via shared variables. He showed that
these problems could not be handled using assignments present in established
languages. He presented his famous scenario of 2 (or more) sequential processes,
each with a critical section, and with the rule that never more than one process
was to execute its critical section. At 1st sight an obvious solution is the
following, using 2 state variables:
\begin{verbatim}[language=Algol]
  // qi means Pi enters its critical section
  VAR q0, q1: BOOLEAN;

  PROCESS P0;
  REPEAT ...
    IF ~q1 THEN q0:=TRUE; CS; q0:=FALSE END;
    ...
  END P0

  PROCESS P1; REPEAT ...
    IF ~q0 THEN q1:=TRUE; CS; q1:=FALSE END;
    ...
  END P1
\end{verbatim}

A long controversy ensued, alternating with new proposals how to solve the
problem and proves that the proposals were wrong. It became clear that a new
instruction was needed that offered a test and an assignment in an atomic fashion.
i.e. without letting another process interfere. Dijkstra postulated his
semaphores with primitives
\begin{verbatim}[language=Algol]
  P(s)  wait until s > 0, then decrement s
  V(s)  increment s
\end{verbatim}

After it became fully recognized that programming was a difficult activity, it
also became clear that programming languages, being the basic, mathematical
formalism, had to be specified with rigor and accuracy. Algol had been the 1st
language defined with a rigorous syntax specification. But its semantics were
less clear, and sometimes even diffuse, certainly not appropriate for the
application of formal, logical proofs.

The 1st step towards formally defined semantics was made by R. Floyd at
Stanford in 1968. He created the notion of \emph{assertion}. These are
predicates involving the program's variables, conditions that were to hold
whenever program execution reached the place where the assertion stood. Every
statement $S$ was to be preceded by an assertion $P$ (precondition) and followed
by an assertion $Q$ (post condition) satisfied after $S$ was executed. It was
thought that $Q$ could then be computed from $P$ and $S$ by a program analyzer
or theorem prover. The semantics (meaning) of a statement $S$ was defined by
the triple $\{P\}\ S\ \{Q\}$. Assignment can now be formally specified by
\begin{verbatim}[language=Algol]
  {P(y)} x := y {P(x)}
\end{verbatim}

Soon afterwards, C.A.R. Hoare published his seminal paper entitled "Axiomatic
definition of programming languages". Here the key idea was that the semantics of
composite statements could be derived from the semantics of their components.
As an example, the semantics of the while statement for repetitions were specified
as follows:

Given $\{P\}\ S\ \{Q\}$, where $P$ is a predicate, then for the while statement
\begin{verbatim}[language=Algol]
  {P} WHILE b DO S END {P & ~b}
\end{verbatim}
holds. $P$ is called its \emph{loop invariant}. The 2nd part of a proof about repetitive
statements must show that progress is guaranteed, that is, that the repetition will
ultimately terminate by invalidating $b$.

E. W. Dijkstra then refined the theory and introduced the notion of predicate
transformer. Given a statement $S$ and a predicate $P$, $S$ would transform $P$ into
$Q$, i.e. $Q = F(P, S)$, where $F$ is a Boolean function. It turned out that it was
more useful to define the transformer as $P = F'(S, Q)$, i.e. to work backwards from
the result to the precondition when proving program correctness: Postulating a desired
result, and then working backwards to find out which states would be acceptable as
starting points of the computation.

Dijkstra was a leading member of the WG 2.1. and an ardent advocate of a scientific,
mathematical, rigorous approach to programming. He was a strong critic of empirical
approaches, of designing by trial and error, and an emphatic advocate of programming
as mathematical engineering. Well-known remains his skepticism about program testing.
He noted that testing can show the presence of errors, but never prove their absence.
Another harsh dictum of his was "Do not give industry what it wants, but what it needs".
He also chided that many people mix up conventional with convenient.

Although it was soon realized that correctness proving was an arduous task, the
influence of this development was strong on the discipline of programming. It was an
incentive to keep languages simple and to proceed in programming always with
correctness concerns in mind. But all his statements did not win Dijkstra admirerers
only.

The problems with correctness proofs are three-fold. 1st, the proofs become equally
long, tedious, and perhaps error-prone as the programs themselves. 2nd, specifications
of a program's pre-condition and post-condition are usually very complex, almost as
complex as the programs themselves, or even worse. 3rd, most programmers lack the
facilities to reason about formal logic to a degree required here. Nevertheless, this
work on correctness proving exerted a distinct influence on how programs are now
developed: With their proof in mind.

Meanwhile, programming, or rather software engineering as it became called, had
become a business of its own, independent of hardware on which programs rested. The
term unbundling, created in legal disputes in the 1970s, had become the way of life.
Companies like Microsoft were the visible result.
