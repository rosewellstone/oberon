\chapter{Procedure Types}
So far, we have regarded procedures exclusively as program parts, i.e. as texts that specify actions
to be performed on variables whose values are numbers, logical values, characters, etc. However,
we may take the view that procedures themselves are objects that can be assigned to variables. In
this light, a procedure declaration appears as a special kind of constant declaration, the value of
this constant being a procedure. If we allow variables in addition to constants, it must be possible to
declare types whose values are procedures. These are called \emph{procedure types}.

A procedure type declaration specifies the number and the types of parameters and, if it is to be a
function procedure, the type of the result. For example, a procedure type with a \verb|REAL| argument
and a result of the same type is declared by
\begin{verbatim}
  Function = PROC(x: REAL): REAL
\end{verbatim}
Mathematical functions, sine, cosine, square root, exponential and logarithm are all of this type.
The general syntax is
\begin{verbatim}
  ProcedureType = PROC [TypeListl]
  TypeList = ([[VAR] Type{, [VAR] Type}])[: returnType]
\end{verbatim}
If we declare a variable \verb|f: Function|, we can then assign \verb|f := Math.sin|.
Thus, the call \verb|f(x)| is equivalent to \verb|Math.sin(x)|.

It is now also possible to declare functions and procedures which themselves have functions
and procedures as parameters. For example, a function procedure to integrate any function
over a certain interval can be expressed in the following way. It adds the values of the
given parametric function over the given interval $a$ to $b$ in $n$ steps.
\begin{verbatim}
  PROC integral(f: Function; a, b: REAL; n: INT): REAL;
    VAR i: INT; sum, dx: REAL;
  BEGIN dx := (b-a)/n; sum := 0;
    FOR i:= 0 TO n-1 DO
      sum := f(a + i*dx + 0.5*dx) + sum
    END;
    RETURN sum * dx
  END integral
\end{verbatim}
The integrations of, for example, the sine function over the interval $0$ to $\pi$, and
that of the exponential function from $0$ to $1$ are expressed simply as
\begin{table}[h!]
  \begin{tabular}{r|c|l}
    integration call                          & yield  & truth \\\hline
    \verb|integral(Math.sin, 0, 3.14159, 20)| & 2.0020 & 2 \\
    \verb|integral(Math.exp, 0, 1      , 20)| & 1.7181 & $e-1$
  \end{tabular}
\end{table}

To conclude, we emphasize the restriction of Oberon:
Procedures assigned to variables or passed as parameters,
\begin{itemize}
  \item must NOT be declared local to any other procedures.
  \item Neither can they be standard procedures (such as \verb|ODD|, \verb|INCL|, etc.).
\end{itemize}
