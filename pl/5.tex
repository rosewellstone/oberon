\chapter{Statements and Expressions}
The specification of an action is called a statement. Statements can be interpreted (executed), and
that interpretation (execution) has an effect. The effect is a transformation of the state of the
computation, the state being represented by the collective values of the program's variables. The most
elementary action is the assignment of a value to a variable. The assignment statement has the form
\begin{verbatim}
  assignment = designator ":=" expression.
\end{verbatim}
and its corresponding action consists of 3 parts in this sequence:
\begin{enumerate}
  \item Evaluate the designator designating a variable.
  \item Evaluate the expression yielding a value.
  \item Replace the value of the variable identified in 1. by the value obtained in 2.
\end{enumerate}
Simple examples of assignments are:
\begin{verbatim}
  i := 1    x := y+z
\end{verbatim}
Here $i$ obtains the value 1, and $x$ the sum of $y$ and $z$, and the previous values are lost.
Evidently, every variable in an expression must previously have been assigned a value. Observe that
the following pairs of statements, executed in sequence, do not have the same effect:
\begin{verbatim}
  i := i+1; j := 2*i
  j := 2*i; i := i+1
\end{verbatim}
Assuming the initial value $i = 0$, the 1st pair yields $i = 1$, $j = 2$, whereas the 2nd pair yields
$j = 0$.  If we wish to exchange the values of variables $i$ and $j$, the statement sequence
\begin{verbatim}
  i := j; j := i
\end{verbatim}
will not have the desired effect. We must introduce a temporary value holder, say $k$, and specify
the 3 consecutive assignments
\begin{verbatim}
  k := i; i := j; j := k
\end{verbatim}
An expression is in general composed of several operands and operators. Its evaluation consists of
applying the operators to the operands in a prescribed sequence, in general taking the operators
from left to right. The operands may be constants, or variables, or functions. (The latter will be
explained in a later chapter.) The identification of a variable will, in general, again require the
evaluation of a designator. Here we shall confine our presentation to simple variables designated
by an identifier. Arithmetic expressions (there exist other expressions too) involve numbers,
numeric variables, and arithmetic operators. These include the basic operations of addition (+),
subtraction (-), multiplication (*), and division. They will be discussed in detail in the chapter on
basic data types. Here it may suffice to mention that the slash (/) is reserved for dividing real
numbers, and that for integers we use the operator DIV which truncates the quotient.

An expression consists of consecutive terms. The expression
\begin{verbatim}
  T0 + T1 + ... + Tn
\end{verbatim}
is equivalent to
\begin{verbatim}
  ((T0 + T1) + ...) + Tn
\end{verbatim}
and is syntactically defined by the rules
\begin{verbatim}
  SimpleExpression =[+|-] term {AddOperator term}.
  AddOperator = + | - | OR
\end{verbatim}

\textbf{Note}: for the time being, the reader may consider the syntactic entities expression and
SimpleExpression as equivalent. Their difference and the operators OR, AND, and NOT will be
explained in the chapter on the data type BOOL.

Each term similarly consists of factors. The term
\begin{verbatim}
  F0 * F1 * ... * Fn
\end{verbatim}
is equivalent to
\begin{verbatim}
  ((F0 * F1) * ... ) * Fn
\end{verbatim}
and is syntactically defined by the rules
\begin{verbatim}
  term = factor {MulOperator factor}.
  MulOperator = * | / | DIV | MOD | &.
\end{verbatim}
Each factor is either a constant, a variable, a function, or an expression itself enclosed by
parentheses.

Examples of (arithmetic) expressions are
\begin{verbatim}
  2*3 + 4*5       = (2*3)+(4*5)   = 26
  15 DIV 4 * 4    = (15 DIV 4)*4  = 12
  15 DIV (4*4)    = 15 DIV 16     = 0
  2 + 3*4 - 5     = 2 + (3*4) - 5 = 9
  6.25/1.25 + 1.5 = 5.0 + 1.5     = 6.5
\end{verbatim}
The syntax of factors, implying that a factor may itself be an expression, is evidently recursive. The
general form of designators will be explained later; here it suffices to know that an identifier
denoting a variable or a constant is a designator.
\begin{verbatim}
  factor = number | string | set | designator
    [ActualParameters] | (expression) | ~factor.
\end{verbatim}
The rules governing expressions are actually quite simple, and complicated expressions are rarely
used. Nevertheless, we must point out a few basic rules that are well worth remembering.
\begin{enumerate}
  \item Every variable in an expression must previously have been assigned a value.
  \item 2 operators must never be written side by side. For instance \verb|a * -b| is illegal and
    must be written as \verb|a*(-b)|.
  \item The multiplication sign must never be omitted when a multiplication is required. For
    example, \verb|2n| is illegal and must be written as \verb|2*n|.
  \item MulOperators are binding more strongly than AddOperators.
  \item When in doubt about evaluation rules (i.e. precedence of operators), use additional
    parentheses to clarify. For example, \verb|a+b*c| may just as well be written as
    \verb|a+(b*c)|.
\end{enumerate}
The assignment is but one of the possible forms of statements. Other forms will be introduced in the
following chapters. We enumerate these forms by the following syntactic definition
\begin{verbatim}
  statement = [ assignment | ProcedureCall | IfStatement
    | WhileStatement | RepeatStatement | ForStatement ].
\end{verbatim}
Several of these forms are structured statements, i.e. some of their components may be statements
again. Hence, the definition of statement is, like that of expressions, recursive.

The most basic structure is the sequence. A computation is a sequence of actions, where each
action is specified by a statement, and is executed after the preceding action is completed. This
strict sequentiality in time is an essential assumption of sequential programming. If a statement
$S_1$ follows $S_0$, then we indicate this sequentiality by a semicolon
\[  S_0; S_1 \]
This statement separator (not terminator) indicates that the action specified by $S_0$ is to be
followed immediately by the action corresponding to $S_1$. A sequence of statements is syntactically
defined as
\begin{verbatim}
  StatementSequence = statement {; statement}.
\end{verbatim}
The syntax of statements implies that a statement may consist of no symbols at all. In this case, the
statement is said to be empty and evidently denotes the null action. This curiosity among statements
has a definite reason: it allows semicolons to be inserted at places where they are actually
superfluous, such as at the end of a statement sequence.
