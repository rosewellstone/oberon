\chapter{Elementary Data Types}
We have previously stated that all variables need be declared. This means that their names are
introduced in the heading of the program. In addition to introducing the name (and thereby enabling
a compiler to detect and indicate misspelled identifiers), declarations have the purpose of
associating a data type with each variable. This data type represents information about the variable
which is permanent in contrast, for example, to its value. This information may again aid in
detecting inconsistencies in an erroneous program, inconsistencies that are detectable by mere
inspection of the program text without requiring its interpretation.

The type of a variable determines its set of possible values and the operations which may be
applied to it. Each variable has a single type that may be deduced from its declaration. Each
operator requires operands of a specific type and produces a result of a specific type. It is therefore
visible from the program text, whether or not a given operator is applicable to a given variable.

Data types may be declared in the program. Such constructed types are usually based on
composition of basic types. There exist a number of most frequently used, elementary types that
are basic to the language and need not be declared. They are called standard types and will be
introduced in this chapter, although some have already appeared in previous examples.

In fact, not only variables have a type, but so do constants, functions, operands and (results of)
operators. In the case of a constant, the type is usually deducible from the constant's notation,
otherwise from its explicit declaration.

First we describe the standard data types of Oberon, and thereafter elaborate on the form of
declarations of variables and constants. Further kinds of data types and declarations are deferred
to later chapters.

\section{INT}
This type represents the whole numbers, and any value of type INT is therefore an integer.
Operators applicable to integers include the basic arithmetic operations
\begin{table}[h!]
  \centering
  \begin{tabular}{c|l}
        & Operations \\\hline
    $+$ & addition \\
    $-$ & subtraction \\
    $*$ & multiplication \\
    DIV & division \\
    MOD & modulus (remainder)
  \end{tabular}
  \caption{Operations on sets}
\end{table}

Integer division is denoted by \verb|DIV|. If we define the integer quotient and the modulus of $x$
and $y$ by \verb|q = x DIV y|, \verb|r = x MOD y|, then $q$ and $r$ are related by the equation
\verb|x = q*y +r| and by the constraint $0 \le r < y$.

For example
\begin{verbatim}
   15 DIV 4 =  3   15 MOD 4 = 3   15 =    3*4+3
  -15 DIV 4 = -4  -15 MOD 4 = 1  -15 = (-4)*4+1
\end{verbatim}
Sign inversion is denoted by the monadic minus sign. Furthermore, there exist the operators
\verb|ABS(x)| and \verb|ODD(x)|, the former yielding the absolute value of $x$, the latter
the boolean result "$x$ is odd".

Every computer will restrict the set of values of type INT to a finite set of integers, usually the
interval $[-2^{N-1}, 2^{N-1}-1]$, where $N$ is a small integer, often 16 or 32, depending on the
number of bits a computer uses to represent an integer. If an arithmetic operation produces a result
that lies outside that interval, then overflow is said to have occured. The computer will give an
appropriate indication and, usually, terminate the computation. The programmer should ensure that
overflows will not result during execution of the program.

\section{REAL}
Values of type REAL are real numbers. The available operators are again the basic arithmetic
operations and ABS. Division is denoted by / (instead of DIV). Constants of type REAL are
characterized by having a decimal point and possibly a decimal scale factor. Examples of real
number denotations are
\begin{verbatim}
  1.5   1.50   1.5E2   2.34E-2   0.0
\end{verbatim}
The scale factor consists of the capital letter \verb|E| followed by an integer. It means that the
preceding real number is to be multiplied by 10 raised to the scale factor. Hence
\begin{verbatim}
  1.5E2 = 150.0,    2.34E-2 = 0.0234
\end{verbatim}
The important point to remember is that real values are internally represented as pairs consisting of
a fractional number and a scale factor. This is called floating-point representation. Of course, both
parts consist of a finite number of digits. As a consequence, the representation is inherently inexact,
and computations involving real values are inexact because each operation may be subject to
truncation or rounding.

The following program makes that inherent imprecision of computations involving operands of type
REAL apparent. It computes the harmonic function
\begin{verbatim}
  H(n) = 1 + 1/2 + 1/3 + ... + 1/n
\end{verbatim}
in 2 different ways, namely once by summing terms from left to right, once from right to left.
According to the rules of arithmetic, the 2 sums ought to be equal. However, if values are truncated
(or even rounded), the sums will differ for sufficiently large $n$. The correct way is evidently
to start with the small terms.
\begin{verbatim}
  PROC Harmonic*;
    VAR i, n: INT; x, d, s1, s2: REAL;
        S: Texts.Scanner;
  BEGIN
    Texts.OpenScanner(S, Oberon.Par.text,
                         Oberon.Par.pos);
    Texts.Scan(S);
    WHILE S.class = Texts.Int DO
      n := S.i; Texts.WriteString(W, "n =");
                Texts.Writelnt(W, n, 6);
      s1 := 0.0; d := 0.0; i := 0;
      REPEAT d := d + 1.0; INC(i);
            s1 :=s1 + 1.0/d;
      UNTIL i >= n;
      Texts.WriteReal(W, s1, 16);
      s2 := 0.0;
      REPEAT s2 := s2 + 1.0/d;
              d :=  d - 1.0; DEC(i)
      UNTIL i = 0;
      Texts.WriteReal(W, s2, 16);
      Texts.WriteLn(W); Texts.Scan(S)
    END;
    Texts.Append(Oberon.Log, W.buf)
  END Harmonic.
\end{verbatim}
The major reason for strictly distinguishing between real numbers and integers lies in the different
representation used internally. Hence, also the arithmetic operations are implemented by
instructions which are distinct for each type. Oberon postulates that expressions with mixed
operands always are of type REAL.

A real number $x$ is transformed into an integer by the standard function \verb|FLOOR(x)|. It yields
the largest integer not greater than $x$.
\begin{verbatim}
  FLOOR(3.1416) = 3   FLOOR(-2.7) = -3   FLOOR(-5.0) = -5
\end{verbatim}

\section{BOOL}
A \verb|BOOL| value is one of the two logical truth values denoted by the symbols \verb|TRUE| and
\verb|FALSE|. Boolean variables are usually denoted by identifiers which are adjectives, the value
\verb|TRUE| implying the presence, \verb|FALSE| the absence of the indicated property. A set of logical
operators is provided which, together with \verb|BOOL| variables, form \verb|BOOL| expressions. These
operators are \& (and), OR, and \~{} (not). Their results are explained as follows:
\begin{verbatim}
  p  & q = "both p and q are TRUE"
  p OR q = "either p or q or both are TRUE"
  ~p = "p is FALSE"
\end{verbatim}
The operators’ exact definition, however, is slightly different, although the results are identical:
\begin{verbatim}
  p  & q = IF p THEN q    ELSE FALSE
  p OR q = IF p THEN TRUE ELSE q
\end{verbatim}
This definition implies that the second operand need not be evaluated, if the result is already known
from the evaluation of the first operand. The notable property of this definition of Boolean
connectives is that their result may be well-defined even in cases where the second operand is
undefined. As a consequence, the order of the operands may be significant.

Sometimes it is possible to simplify Boolean expressions by application of simple transformation
rules. A particularly useful rule is de Morgan's law stating the equivalences
\begin{verbatim}
  ~p  & ~q = ~(p OR q)
  ~p OR ~q = ~(p  & q)
\end{verbatim}
Relations produce a result of type BOOL, i.e. TRUE if the relation is satisfied, FALSE if not.
For example
\begin{verbatim}
  7 = 12       FALSE
  7 < 12       TRUE
  1.5 >= 1.6   FALSE
\end{verbatim}
Relations are syntactically classified as expressions, and the two comparands are so-called
SimpleExpressions (see also chapter on expressions and statements). The result of a comparison
is of type BOOL and can be used in control structures such as if, while, and repeat statements.
The symbol \# stands for unequal.
\begin{verbatim}
  expr = exp [rel exp]
  rel  = = | # | < | <= | > | >= | IN
\end{verbatim}
It should be noted that, similar to arithmetic operators, there exists a precedence hierarchy among
the Boolean operators. \~{} has the highest precedence, then follows \&, then OR, and last are the
relations. As with arithmetic expressions, parentheses may be used freely to make the association
of operators explicit. Examples of Boolean expressions are
\begin{verbatim}
  x = y
  (x <= y) & (y <Z)
  (x > y) OR (y >= Z)
  ~p OR q
\end{verbatim}
Note that a construct such as \verb|x<y & z<w| is illegal.

The example above draws attention to the rule that the operands of an operator (including relational
operators) must be of the same type. The following relations are therefore illegal:
\begin{verbatim}
    1   = TRUE
    5   = 5.0
  i + j = p OR q
\end{verbatim}
Also incorrect is e.g. \verb|x <= y < z|, which must be expanded into \verb|(x <= y) & (y < z)|.
However, the following are correct Boolean expressions:
\begin{verbatim}
  it < k - m
  p OR q = (i < j)
\end{verbatim}
A final hint: although "p = TRUE" is legal, it is considered poor style and better expressed as p.
Similarly, replace "p = FALSE" by "~p".

\section{CHAR}
Every computer system communicates with its environment via some input and output devices.
They read, write, or print elements taken from a fixed set of characters. This set constitutes the
value range of the type CHAR. Unfortunately, different brands of computers may use different
character sets, which makes communication between them (i.e. the exchange of programs and
data) difficult and often tedious. However, there exist internationally standardised sets. Here we
refer to ISO’s Latin-1 set.

The Latin-1 standard defines a set of 128 characters, 33 of which are so-called (non-printing)
control characters. The remaining 95 elements are visible printing characters shown in the following
table. The set is ordered, and each character has a fixed position or ordinal number. For example,
A is the 66th character and has the ordinal number 65. The ISO standard, however, leaves a few
places open, and they can be filled with different characters according to national desires to
establish national standards. Most widely used is the American standard, also called ASCII
(American Standard Code for Information Interchange). Here we tabulate the ASCII set.
\begin{table}[h!]
  \centering
  \begin{tabular}{r c c c c c c c c}
      &  0  &  1  &  2 & 3 & 4 &              5 & 6 &  7 \\
    0 & nul & dle &    & 0 & @ &              P & ` &  p \\
    1 & soh & dc1 &  ! & 1 & A &              Q & a &  q \\
    2 & stx & dc2 &  " & 2 & B &              R & b &  r \\
    3 & etx & dc3 & \# & 3 & C &              S & c &  s \\
    4 & eot & dc4 & \$ & 4 & D &              T & d &  t \\
    5 & enq & nak & \% & 5 & E &              U & e &  u \\
    6 & ack & syn & \& & 6 & F &              V & f &  v \\
    7 & bel & etb &  ' & 7 & G &              W & g &  w \\
    8 & bs  & can &  ( & 8 & H &              X & h &  x \\
    9 & ht  & em  &  ) & 9 & I &              Y & i &  y \\
    A & lf  & sub &  * & : & J &              Z & j &  z \\
    B & vt  & esc &  + & ; & K &              [ & k & \{ \\
    C & ff  & fs  &  , & < & L & \textbackslash & l &  | \\
    D & cr  & gs  &  - & = & M &              ] & m & \} \\
    E & so  & rs  &  . & > & N &           \^{} & n & \~{} \\
    F & si  & us  &  / & ? & O &             \_ & o & del
  \end{tabular}
  \caption{Table of ASCII characters}
\end{table}
The ordinal number of a character is obtained by taking its row number and adding its column number. These
numbers are customarily given in hexadecimal form, and we follow this habit here too. The first two
columns contain the control characters; they are customarily denoted by abbreviations hinting at
their intended meaning. However, this meaning is not inherent in the character code, but only
defined by its interpretation. At this point it suffices to remember that these characters are (usually)
not printable.

Constants of type CHAR are denoted by the character being enclosed in quote marks (single-
character string). They can be assigned to variables of type CHAR. These values cannot be used in
arithmetic operations. Arithmetic operators can be applied, however, to their ordinal numbers
obtained from the transfer function ORD(c). Inversely the character with the ordinal number n is
obtained with the transfer function CHR(n). These two complementary functions are related by the
equations
\begin{verbatim}
  CHR(ORD(c)) = c and ORD(CHR(n)) =n
\end{verbatim}
for $0 \le n< 128$. They permit to compute the numeric value represented by the digit $c$ as
\begin{verbatim}
  ORD(c) - ORD("0")
\end{verbatim}
and to compute the digit representing the numeric value $n$ as
\begin{verbatim}
  CHR(n + ORD("0"))
\end{verbatim}
These 2 formulas depend on the adjacency of the 10 digits in the ISO character set, where
\verb|ORD("0") = 30H = 48|. They are typically used in routines converting sequences of digits into
numbers and vice-versa. The following program piece reads digits and assigns the decimally
represented number to a variable $x$.
\begin{verbatim}
  x := 0; Texts.Read(R, c);
  WHILE ("0" <= c) & (c <= "9") DO
    x := 10*x + (ORD(c) - ORD('0"));
    Texts.Read(R, c)
  END
\end{verbatim}
Control characters are used for various purposes, mainly to control the functions of devices, but
also to delineate and structure text. An important role of control characters is to specify the end of a
line or of a page of text. There is no universally accepted standard for this purpose. We shall
denote the control character signifying a line end by the identifier EOL (end of line); its actual value
depends on the system used.

In order to be able to denote non-printable characters, Oberon uses their hexadecimal ordinal
number followed by the capital letter \verb|X|. For example, \verb|ODX| is the value of type \verb|CHAR|
denoting the control character \verb|cr| (carriage return, line end) with ordinal number 13.

\section{SET}
The values which belong to the type SET are sets of integers between 0 and N-1, where N is a
constant defined by the computer system used. It is usually the computer's word length or a small
multiple of it, typically 32. Constants of this type are denoted as sets. Examples are
\begin{verbatim}
  {2, 3, 5, 7, 11} {0} {8, 15}
\end{verbatim}
The notation \verb|m..n| is a shorthand for \verb|m, m+1,... , n-1, n|.
\begin{verbatim}
  set = [qualident] "{" [element{, element}] "}"
  element = expression [.. expression]
\end{verbatim}
\begin{table}[h!]
  \centering
  \begin{tabular}{l r|c}
    \multicolumn{2}{r|}{Operations on sets $u$ and $v$} & Set of elements that are in\\\hline
    $+$ &                union $u+v$ & either $u$ or $v$ \\
    $-$ &           difference $u-v$ &   $u$ but not $v$ \\
    $*$ &         intersection $u*v$ &  both $u$ and $v$ \\
    $/$ & symmetric difference $u/v$ & either $u$ or $v$, but not both
  \end{tabular}
  \caption{Operations on sets}
\end{table}
The membership operator \verb|IN| is regarded as a relational operator. The expression \verb|i IN u|
is of type \verb|BOOL|. It is \verb|TRUE|, if $i$ is a member of the set $u$. Sets are represented in
computer systems as sets of bits, i.e. by the characteristic function of the set. The $i$-th bit of $u$,
for example, is 1, if $i$ is a member of $u$, 0 otherwise. Hence, the set operators are implemented as
logical operations applied to the $N$ members of the set variable. They are therefore very efficient and
their execution time is usually even less than that of an addition of integers.
