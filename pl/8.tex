\chapter{Constant \& Variable Declarations}
It has been mentioned previously that all identifiers used in a program must be declared
in the program's heading, unless they are standard identifiers known in every program (or
are imported from some other module).

If an identifier is to denote a constant value, it must be introduced by a constant
declaration which indicates the value for which the constant identifier stands. A
constant declaration has the form:
\begin{verbatim}
  ConstantDeclaration = identifier = ConstExpression
  ConstExpression = expression
\end{verbatim}
A \verb|ConstExpression| is an expression containing constants only. More precisely, its
evaluation must be possible by a \emph{mere textual scan} without execution of the program.
A sequence of constant declarations is preceded by the symbol \verb|CONST|. Example:
\begin{verbatim}
  CONST N = 16;
      EOL = ODX;
    empty = {};
        M = N - 1;
\end{verbatim}
Constants with explicit names aid in making a program readable, provided that the constants
are given suggestive names. If, e.g., the identifier $N$ is used instead of its value
throughout a program, a change of that constant can be achieved by changing the program
in a single place only, namely in the declaration of $N$. This avoids the common mistake
that some instances of the constant, spread over the entire program text, remain undetected
and therefore are not updated, leading to inconsistencies.

A variable declaration looks similar to a constant declaration. The place of the constant's
value is taken by the variable's type which, in a sense, can be regarded as the variable's
constant property.  Instead of an equal sign, a colon is used.
\begin{verbatim}
  VariableDeclaration = IdentList: type
  IdentList = identifier {, identifier}
\end{verbatim}
Variables of the same type may be listed in the same declaration, and a sequence of
declarations is preceded by the symbol \verb|VAR|. Example:
\begin{verbatim}
  VAR i, j, k: INT;
      x, y, Z: REAL;
           ch: CHAR
\end{verbatim}
