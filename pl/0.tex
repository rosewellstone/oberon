\setcounter{chapter}{-1}
\chapter{Preface}

This text is an introduction to programming in general, and a guide to programming in the language
Oberon in particular. It is primarily oriented towards people who have already acquired some basic
knowledge of programming and would like to deepen their understanding in a more structured way.
Nevertheless, an introductory chapter is added for the benefit of the beginner, displaying in a
concise form some of the fundamental concepts of computers and of programming them. The text
is therefore also suitable as a self-contained tutorial. The notation used is Oberon, which lends itself
well for a structured approach and leads the student to a working style that has generally become
known under the heading of structured programming.

As a manual for programming in Oberon, the text covers (almost) all facilities of this language. Part
1 covers the basic notions of variable, expression, assignment, conditional and repeated statement,
and the data structure of the indexable array. Together with Part 2, which introduces the important
concept of the procedure or subroutine, it contains essentially the material commonly discussed in
introductory programming courses. Part 2 also presents data types and structures and constitutes
the core of an advanced course on programming. Part 3 introduces the notion of a module, a
concept that is fundamental in the design of large programming systems, and to programming in
teams. The two basic notions of files and texts are introduced in the forms of modules containing
the operations performed on files and texts. Texts are presented as our standard medium for
program input and output.

The language Oberon, published in 1988, has a long history. It is a descendant of Algol 60 (1960),
Pascal (1970), and Modula (1979). Algol 60 [1] was designed by an international committee of 13
scientists, and it was the first language to be specified with a rigorous formalism, a syntax, in
machine-independent form. Algol was largely oriented toward numerical algorithms, as were
computers at that time. Pascal [2] was the result of an enduring debate about widening the range of
application of Algol, and it became widely used, in particular but not only, in education. Modula-2 [3]
introduced the concept of modules, the parts of large systems connected by clearly specified
interfaces. The module allows to hide details of procedures and variables from its users (clients),
and it embodies the principle of information hiding. The language Oberon [4] emerged from the
urge to reduce the complexity of programming languages, of Modula in particular. This effort
resulted in a remarkably concise language. The extent of Oberon, the number of its features and
constructs, is smaller even than that of Pascal. Yet it is considerably more powerful.

The one feature that was added in Oberon was the extensibility of data types (record types).
Whereas in strongly types languages, such as Algol, Pascal, and Modula, every constant, variable,
or function has a fixed type, recognizable from the program text, Oberon allows to define
hierarchies of types, and to determine the actual type of a variable (within the hierarchy) at run-
time. This, together with records containing fields of procedural types, is the stem of object-oriented
programming. Such records are then called objects, and the procedural fields are called methods.
Here, we do not touch upon this subject.

~~~~~~~~~~~~~~~~~~~~~~~~~~~~~~~~~~~~~~~~~Zurich, Oct. 1, 2004 N.W.

\section*{Reference}
\begin{enumerate}
  \item P. Naur, Ed. \emph{Revised Report on the Algorithmic Language Algol 60.} Comp. J. 5, 349-367 (1962), and Comm. ACM, 6 (1963) 1-17.
  \item N. Wirth. \emph{The Programming Language Pascal.} Acta Informatica, 1 (1971), 35-63.
  \item N. Wirth. \emph{Programming in Modula-2.} Springer-Verlag, 1982. ISBN 0-387-50150-9
  \item N. Wirth. \emph{The Programming Language Oberon.} Software - Practice and Experience, 18, 7, (July 1988), 671-690.
\end{enumerate}
