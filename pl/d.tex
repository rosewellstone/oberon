\chapter{Function Procedures}
So far we have encountered 2 possibilities to pass a result from a procedure body to its calling
place: the result is either assigned to a non-local variable or to a variable parameter. There
exists a 3rd method: the function procedure. It permits the use of the computed result (as an
intermediate value) in an expression. The function procedure identifier stands for a computation
as well as for the computed result. The procedure declaration is characterized by the indication
of the type of the result following the parameter list. As an example, we rephrase the power
computation given above.
\begin{verbatim}
  PROC power(x: REAL; i: INT): REAL;
    VAR z: REAL;
  BEGIN z := 1.0:
    WHILE i > 0 DO
      IF ODD(i) THEN z := z*x END;
      x := x*x; i := 1 DIV 2
    END;
    RETURN z
  END power
\end{verbatim}
Possible calls are
\begin{verbatim}
  u    := power(2.5, 3)
  A[i] := power(B[i], 2)
  u    := x + power(y, i+1) / power(z, i-1)
\end{verbatim}
The result of the function evaluation is specified as an expression following the symbol
\verb|RETURN| and immediately preceding the symbpl \verb|END|.

Calls inside an expression are called \emph{function designators}. Their syntax is the same
as that of procedure calls. However, a parameter list is mandatory, although it may be empty.
\begin{verbatim}
  ReturnStatement = "RETURN" [expression].
\end{verbatim}
We now revisit the previous example of adding the elements of an array and formulate it as
a function procedure:
\begin{verbatim}
  PROC sum(VAR a: Vector; n: INT): REAL;
    VAR i: INT; s: REAL;
  BEGIN s := 0.0:
    FOR i := 0 TO n-1 DO s := a[i]+s END;
    RETURN s
  END sum
\end{verbatim}
This procedure, as previously specified, sums the elements $a[0] \dots a[n-1]$, where $n$
is given as a value parameter, and may be different from (but not larger than!) the number
of elements $N$. A more elegant solution specifies $a$ as an open array, omitting the
explicit indication of the array's size.
\begin{verbatim}
  PROC sum(VAR x: ARRAY OF REAL): REAL;
    VAR i: INT; s: REAL;
  BEGIN s := 0.0:
    FOR i := 0 TOLEN(x)-1 DO's := x{i] +s END;
    RETURN s
  END sum
\end{verbatim}
Obviously, procedures are capable of generating more than one result by making assignments
to several variables. Only one value, however, can be returned as the result of a function.
This value, moreover, cannot be of a structured type. Therefore, the other results must be
passed to the caller via VAR parameter or assignment to variables that are not local to the
function procedure. Consider for example the following procedure which computes a primary
result denoted as the function's value and a secondary result used to count the number of
times the procedure is called.
\begin{verbatim}
  PROC square(x: INT): INT;
  BEGIN INC(n); RETURN x*x
  END square
\end{verbatim}
There is nothing remarkable about this example as long as the secondary result is used for
its indicated purpose. However, it might be misused as follows:
\begin{verbatim}
  m := square(m) +n
\end{verbatim}
Here the secondary result occurs as an argument of the expression containing the function
designator itself. The consequence is that e.g. the values \verb|square(m)+n| and
\verb|n+square(m)| differ, seemingly defying the basic law of commutativity of addition.

Assignments of values from within function procedures to non-local variables are called
\emph{side-effects}. The programmer should be fully aware of their capability of producing
unexpected results when the function is used inappropriately. We summarize:
\begin{enumerate}
  \item A function procedure specifies a result which is used at its place of call as an
    argument of an expression.
  \item The result of a function procedure cannot be structured.
  \item If a function procedure generates secondary results, it is said to have side-effects.
    These must be used with care. It is advisable to use a regular procedure instead,
    which passes its results via \verb|VAR| parameters.
  \item We recommend to choose function identifiers which are nouns rather than verbs. The
    noun then denotes the function's result. Boolean functions are appropriately labelled
    by an adjective.  In contrast, regular procedures should be designated by a verb
    describing their action.
\end{enumerate}
