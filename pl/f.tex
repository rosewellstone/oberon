\chapter{Type Declarations}
Every variable declaration specifies the variable's type as its constant property. The type can be
one of the standard, primitive types, or it may be of a type declared in the program itself. Type
declarations have the form
\begin{verbatim}
  TypeDeclaration = identifier "=" type.
\end{verbatim}
They are preceded by the symbol TYPE. Types are classified into unstructured and structured
types. Each type essentially defines the set of values which a variable of this type may assume. A
value of an unstructured type is an atomic unit, whereas a value of structured type has components
(elements). For example, the type INT is unstructured; its elements are atomic. It does not
make sense, e.g. to refer to the third bit of the value 13; the circumstance that a number may "have
a third bit", or a second digit, is a characteristic of its (internal) representation, which intentionally is
to remain unknown.

In the following sections we shall show how to declare structured types. We distinguish between
various structuring methods of which we have so far encountered the array only. In addition, there
exists the record type. A facility to introduce structures that vary dynamically during program
execution is based on the concept of pointers and will be discussed in a separate chapter.
\begin{verbatim}
  type = qualident | ArrayType | RecordType
              | PointerType | ProcedureType.
\end{verbatim}
Before proceeding to the various kinds of types, we note that in general, if a type T is declared by
the declaration
\begin{verbatim}
  TYPE T = someType
\end{verbatim}
and a variable t is declared as
\begin{verbatim}
  VAR t: T
\end{verbatim}
then these two declarations can always be merged into the single declaration
\begin{verbatim}
  VAR t: someType
\end{verbatim}
However, in this case fs type has no explicit name and therefore remains anonymous. Typically,
record types are given explicit names.

The concept of type is important, because it divides a program's set of variables into disjoint
classes. Inadvertent assignments among members of different classes can therefore be detected
by a mere inspection of the program text without executing the program. Given, for example, the
declarations
\begin{verbatim}
  VAR b: BOOL; i: INT; x: REAL
\end{verbatim}
the assignments \verb|b := i| and \verb|i := x| are illegal, because the types of the variables
are incompatible.
