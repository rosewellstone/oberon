\chapter{Representation of Oberon Programs}
The preceding chapter has introduced a formalism, by which the structures of well-formed
programs will subsequently be defined. It defines, however, merely the way in which programs are
composed as sequences of symbols, in contrast to sequences of characters. This "shortcoming" is
quite intentional: the representation of symbols (and thereby programs) in terms of characters is
considered too much dependent on individual implementations for the general level of abstraction
appropriate for a language definition. The creation of an intermediate level of representation by
symbol sequences provides a useful decoupling between language and ultimate program
representation. The latter depends on the available character set. As a consequence, we need to
postulate a set of rules governing the representation of symbols as character sequences. The
symbols of the Oberon vocabulary are divided into the following classes:
\begin{enumerate}
  \item identifiers,
  \item numbers,
  \item strings,
  \item operators and delimiters, and
  \item comments.
\end{enumerate}

The rules governing their representation in terms of the standard ISO character set are the following:
\begin{enumerate}
  \item Identifiers are sequences of letters and digits, the first of which must be a letter.
  They are case-sensitive.
  \begin{verbatim}
    id = letter {letter|digit}.
  \end{verbatim}
  Examples of well-formed identifiers are:
  \begin{verbatim}
    Alice likely jump BlackBird SR71
  \end{verbatim}
  Examples of invalid identifiers (and the reason):
  \begin{table}[h!]
    \centering
    \begin{tabular}{l l}
      \verb|sound proof| & (blank space is not allowed) \\
      \verb|sound-proof| & (neither is a hyphen) \\
      \verb|Miller's|    & (no apostrophe allowed) \\
      \verb|2N|          & (1st character must be a letter)
    \end{tabular}
  \end{table}

  Sometimes an identifier has to be qualified by another identifier; this is expressed by
  prefixing $i$ with $j$ and a period ($j.i$); the combined identifier is called a \emph{
  qualified identifier} (abbreviated as \verb|qid|).  Its syntax is
  \begin{verbatim}
    qid = {id "."} id
  \end{verbatim}

  \item Numbers are either integers or real numbers. The former are denoted by sequences of digits.
  Numbers must not include any spaces. Real numbers contain a decimal point and a fractional part.
  In addition, a scale factor may be appended. It is specified by the letter \verb|E| and an integer
  which is possibly preceded by a sign. The \verb|E| is pronounced as "times 10 to the power of".

  Examples of well-formed numbers are:
  \begin{verbatim}
    1981 1 3.25 51E3 4.0E-10
  \end{verbatim}
  Examples of character sequences that are not recognized as numbers (and the reason):
  \begin{table}[h!]
    \centering
    \begin{tabular}{l l}
      \verb|1,5|       & (no comma may appear) \\
      \verb|1'000'000| & (neither may apostrophs) \\
      \verb|3.5En|     & (no letters allowed, except E)
    \end{tabular}
  \end{table}

  The exact rules for forming numbers are given by the following syntax:
  \begin{verbatim}
    number = int | real
    int    = dec | hex H
    real   = [sign]dec.{digit}[E[sign]dec]
    sign   = + | -
    dec    = digit {digit}
    hex    = digit {hexdig}
    hexdig = digit | A-F
  \end{verbatim}
  \textbf{Note}: Integers are taken as octal numbers, if followed by the letter \verb|E|,
  or as hexadecimal numbers, if followed by the letter \verb|H|.

  \item Strings are sequences of any characters enclosed in quote marks. In order that the closing
  quote is recognized unambiguously, the string itself evidently cannot contain a quote mark.

  Examples of strings are:
  \begin{verbatim}
    "no comment"    '""'    "Buck's Corner"
  \end{verbatim}

  \item Operators and delimiters are either special characters or reserved words. Those
  composed of special characters are:
  \begin{table}[h!]
    \begin{tabular}{r l}
                          + & addition, set union \\
                          - & subtraction, set difference \\
      \textasteriskcentered & multiplication, set intersection \\
                          / & division, symmetric set difference \\
                         := & assignment \\
                         \& & logical AND \\
                       \~{} & logical NOT \\
                          = & equal \\
                         \# & unequal \\
                          < & less than \\
                          > & greater than \\
                         <= & less than or equal \\
                         >= & greater than or equal \\
                        ( ) & parentheses \\
                       ~[~] & index brackets \\
                      \{ \} & set braces \\
                      (* *) & comment brackets \\
           \textasciicircum & dereferencing operator \\
        , . ; : .. \textbar & punctuation symbols
    \end{tabular}
  \end{table}

  Reserved words are written in upper-case letters and must not be used as identifiers.
  Hence it is advantageous to memorize the following enumeration list, whose meaning
  will be explained throughout subsequent chapters:
  \begin{verbatim}
    BY    DIV    CASE    ARRAY    IMPORT
    DO    END    ELSE    BEGIN    MODULE
    IF    FOR    PROC    CONST    RECORD
    IN    MOD    THEN    ELSIF    REPEAT
    IS    SET    TRUE    FALSE    RETURN
    OF    VAR    TYPE    UNTIL    POINTER
    OR    TO             WHILE
  \end{verbatim}

  It is customary to separate consecutive symbols by a space, i.e. one or several blanks.
  However, this is mandatory only in those cases where the lack of such a space would merge
  the two symbols into one. For example, in "\verb|IF x = y THEN|" spaces are necessary
  in front of $x$ and after $y$, but could be omitted around the equal sign.

  \item Comments may be inserted between any 2 symbols. They are arbitrary sequences of
  characters enclosed in the comment brackets (* and *). Comments are skipped by compilers
  and serve as help info to us human beings. They might also serve to signal instructions
  (options) to the compiler.
\end{enumerate}
