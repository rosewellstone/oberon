\chapter{Modules}
\label{ch:mod}
Modules are the most important feature distinguishing Oberon from its ancestor Pascal. We have
already encountered modules, simply because every program is a module. However, most systems
are not only a single module, but rather a set of several modules, each module containing declared
objects such as constants, variables, procedures, and types. Objects declared in a module \verb|M0|
can be referenced in another module \verb|M1|, if they are explicitly made to be known in \verb|M1|,
i.e. if they are imported by \verb|M1|. In the examples of the preceding chapters, we have typically
imported procedures for input and output from modules containing collections of frequently used
procedures. These procedures are actually part of our program, even if we have not programmed them
and they are textually disjoint.

The key point is that modules can be kept in a program \emph{library} and are automatically
referenced when a programmer's program is loaded and executed. Hence it is possible to prepare
collections of frequently used operations (such as for input and output), and to avoid
reprogramming them each time a program needs such operations. Modern implementations go even
one step further and offer what is called separate compilation. This signifies that such modules
are not stored in the program library in source form as Oberon texts, but rather in compiled
form. Upon program loading, the compiled (main) module is joined (linked) with the precompiled
modules from which it imports objects. In this case, the compiler must also have access to
descriptions of the objects of the previously compiled, imported modules, when the importing
program is compiled. This facility distinguishes separate compilation from independent
compilation as it exists in typical implementations of Fortran, Pascal, and assemblers.

Every subsidiary module may again import objects from other modules. A program therefore
constitutes an entire hierarchy of modules. The main program is said to be at the highest level,
those modules which do not import objects at all being at the lowest level. Usually, a programmer is
not even aware of this hierarchy, because his modules import objects from modules that he has not
programmed himself; therefore he is unaware of their imports and of the module hierarchy below
them. In principle, however, his program is the text written by himself, extended with the texts of the
imported modules.

These extensions are usually quite large (even if the direct imports consist of a few output
procedures only). In principle, the indirect imports constitute the entire environment or OS.
In a single-user computer system, there is virtually no need for any parts that are neither
directly nor indirectly imported by the main program. However, some modules, such as the
basic I/O and file system, may be required by all programs and therefore become de facto
resident and may therefore be regarded as the OS.

The principal motivation behind the partitioning of a program into modules is - beside the use of
modules provided by other programmers - the establishment of a hierarchy of abstractions. For
example, in the previously encountered cases of imported I/O procedures, we merely wish
to have them available, but do not need to know - or rather do not wish to bother to learn - how
these procedures function in detail. To abstract means to "take away" from the essentials and
thereby to ignore certain details. Each module constitutes an abstraction, if we regard it from "the
outside". We even wish to go one step further: we wish not only to ignore the details of its innards,
but to hide them. The primary reason for this wish is that if the innards are protected from outside
access, we can guarantee their correct functioning, thereby being able to limit the area of error
search in the case of a malfunctioning program.

The 2nd, but not less important reason is to make it possible to change (improve) the innards of
imported modules without having to change (and/or recompile) the importing modules. This effective
\emph{decoupling} of modules is indispensible for the development of large programs, in particular,
if modules are developed by different people, and if we regard the OS as the low section of a
program's module hierarchy. Without decoupling, any change or correction in an OS or in library
modules would become virtually impossible.

A direct consequence of this need for decoupling is the necessity for a textual separation of the
essentials from the details. The essentials of a module are the properties of objects that are
importable into other modules; the details are those parts that are to be hidden and protected.
In Oberon, objects to be visible in other modules are specially marked with an asterisk
following their decalaration identifiers.

An importer of a module needs to know the properties of those objects only which are exported by
the imported module. It is customary and convenient to provide an extract of the module containing
those relevant declarations only. We call it the definition of a module. The definition forms a
contract between the client (importer) and the designer of a module. It is therefore also called the
\emph{interface} of the module. The implementation part remains the property of the module's designer.
As long as the designer alters details without affecting the definition, he need not report his
activity to the clients of his module. Modules are compiled separately, and are therefore called
\emph{compilation units}.

Concluding this introduction to the module concept, we postulate that adequate implementations
provide full type compatibility checking between objects, independent of whether these objects are
declared in the same or in different modules, i.e. the checking mechanism of the compiler functions
also across module boundaries. However, the programmer must realize that this checking provides
no absolute safeguard against mistakes. After all, it concerns formal, syntactic aspects only; it
does not cover the \emph{semantics}. It would not detect, for instance, the replacement of the
algorithm for the sine function by that of the cosine function. However, it must not be regarded
as the duty of a compiler to protect programmers against malicious "colleagues".
