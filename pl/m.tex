\chapter{The Concept of a Sequence}\label{ch:seq}
\section{About IO}
The usefulness and the success of high-level programming languages rests on the principle
of abstraction, the hiding of details that pertain to the computer which is used to execute
the program rather than to the algorithm expressed by the program. The domain that has most
persistently relied on abstraction is that of IO operations. This is not
surprising, because IO inherently involve the activation of devices that are
peripheral to the computer, and whose structure, function, and operation differ strongly
among various kinds and brands. Many programming languages have typically incorporated
statements for reading and writing data in sequential form without reference to specific
devices or storage media. Such an abstraction has many advantages, but there exist always
applications where some property of some device is to be utilized that is poorly, if at all,
available through the standard statements of the language. Also, generality is usually
costly, and consequently operations that are conveniently implemented for some devices may
be inefficient if applied to other devices. Hence, there also exists a genuine desire to
make transparent the properties of some devices for applications that require their efficient
utilization. Simplification and generalization by suppression of details is then in direct
conflict with the desire for transparency for efficient use.

Oberon has resolved (or rather circumvented) this intrinsic dilemma, by not including any
IO statements at all. This somehow extreme approach is acceptable for 2 facilities:
\begin{itemize}
  \item[$1^{st}$,] there is the module structure allowing the construction of a hierarchy of
    (library) modules representing various levels of abstraction.
  \item[$2^{nd}$,] Oberon permits the expression of computer specific operations, such as
    communication with peripheral interfaces.
\end{itemize}
These operations are typically contained in modules at the lowest level of this hierarchy,
and are therefore counted among the so-called low-level facilities.
\begin{itemize}
  \item A program wishing to ignore the details of device handling imports procedures
    from the standard modules at the higher levels of this hierarchy.
  \item When desiring utmost efficiency or requiring access to specific properties of
    specific devices, one either uses low-level modules, so-called device drivers, or
    uses the primitives themselves.
\end{itemize}
In the latter case, the programmer pays the price of intransportability, for his programs
refer directly to particulars of either his computer or its OS.

It is impossible to present in this context any operations of devices at the low levels of
the module hierarchy, because there exists a wide variety of such devices. We restrict the
following material to the presentation of 2 typical modules used in performing conventional
IO operations. Module \verb|Texts| we have already encountered in examples in preceding
chapters. The main concept to be presented is the file, data considered as sequences of
elements of the same type. For this purpose, we present the definition of module \verb|Files|.

We must generally distinguish between legible and illegible IO. Legible IO serves to
communicate between the computer and its user. Mostly the data elements are characters i.e.
values of type \verb|CHAR|; the exception is graphical IO. Legible data are input through
keyboards, scanners, etc. Legible output is generated by displays and printers. Illegible
IO is made from and to so-called peripheral storage media, such as disks and tapes, but
also may originate from sensors - e.g. in laboratories or drawing offices - and to devices
that are controlled by computers, such as plotters, factory assembly lines, traffic signals,
and networks. Data for illegible IO can be of any type, and need not be of type \verb|CHAR|.

The vast majority of IO operations of both the legible and illegible variety is appropriately
considered as \emph{sequential}. Their data are of a structure that does not exist as a
basic data structuring method in Oberon, such as the array or the record. Sequences have
the following characteristics:
\begin{enumerate}
  \item All elements are of the same type.
  \item The number of elements (called the \emph{length} of the sequence) is not known a
    priori. The sequence is therefore (a simple case of) a dynamic structure.
  \item A sequence can be modified only by appending elements at its end, called \emph{writing}.
  \item Only a single element of a stream is visible (accessible) at any one time. Accessing
    this element is called \emph{reading}, and the sequence is read by advancing from one
    element to the next.
\end{enumerate}
In passing we state that the sequence as described above is perhaps the most successful
case of data abstraction encountered. It is certainly more widely used in actual practice
than the often quoted examples of stacks and queues. The language Pascal had included it
among its basic data structuring methods along with arrays, records, and sets.

Before proceeding with the postulation and explanation of modules for handling sequences,
which we will call \emph{files} and \emph{texts}, we wish to point out an important
separation of function performed for legible IO.
\begin{itemize}
  \item[] On the one hand, there is the actual transport of data to and from the computer.
    This involves the activation and sensing of the state of the peripheral device, be it
    a keyboard, a display, or a printer.
  \item[] On the other hand, there is the function of transforming the representation of
    data.
\end{itemize}
If, for example, the value of an expression of type \verb|INT| is to be transmitted to a
display, the computer-internal representation must be transformed into the decimal
representation as a sequence of digits. The display device then translates the character
representation (usually consisting of 8 bits for each character) into a pattern of visible
dots or lines. However, the former translation can be considered as device independent,
and is therefore a prime candidate for separation from device specific operations. It can
be performed by the same routines without regard whether the sequence is to be stored on
a disk or to be made visible on a display.

A 3rd class of functions that can well be separated from physical data transfer and from
conversion of data representation, pertains to devices associated with more than a single
sequence, the primary example being a disk store. We refer to the operations of allocating
storage space and associating names with individual texts or files. Considering that texts
and files are dynamic structures, storage allocation is of considerable complexity. The
naming of individual files and in particular the management of directories (in order to
quickly locate individual files) is another task requiring an elaborate mechanism. Both
storage allocation and directory management are the tasks of individual service modules.
There seem to be as many ways to manage these tasks as there are OSes. And this is precisely
where diversity transcends the many levels of IO operations. We therefore strictly adhere
to and confine ourselves to the abstract notion of a sequence.

\section{Files and Riders}
Elements of a sequence cannot be identified by an index, as in the case of arrays, nor by
a field name, as in the case of a record. Elements are instead accessed one by one,
advancing strictly sequentially. This notion itself implies the postulation of an access
machanism. In the Oberon module \verb|Files|, we call it \verb|Rider|. A rider is a data
structure, an object, which can be placed on a file at a given position, and then be used
to write or read single elements. Hence, \verb|Files| defines not only 1, but 2 types:
\begin{itemize}
  \item[$1^{st}$,] \verb|File| stands for the data, and
  \item[$2^{nd}$,] \verb|Rider|, for the operations performed.
\end{itemize}
It now follows, that several riders may be positioned on a same file and be moved independently.
Not the file but each rider has a reading/writing position. However, normally only 1 is used.
\begin{verbatim}
DEFINITION Files;
  TYPE File;
  Rider = RECORD eof: BOOL; res: LONGINT END;

  PROC Old(name: ARRAY OF CHAR): File;
  PROC New(name: ARRAY OF CHAR): File;
  PROC Register(f: File);
  PROC Close   (f: File);
  PROC Purge   (f: File);
  PROC Length  (f: File): LONGINT;
  PROC GetDate (f: File; VAR t, d: LONGINT);

  PROC Set (VAR r: Rider; f: File; pos: LONGINT);
  PROC Pos (VAR r: Rider): LONGINT;
  PROC Base(VAR r: Rider): File   ;
  PROC Read      (VAR r: Rider; VAR c: CHAR    );
  PROC Readint   (VAR r: Rider; VAR x: INT     );
  PROC ReadReal  (VAR r: Rider; VAR x: REAL    ):
  PROC ReadLReal (VAR r: Rider; VAR x: LONGREAL);
  PROC ReadString(VAR r: Rider; VAR x: ARRAY OF CHAR);

  PROC Write      (VAR r: Rider; c: CHAR);
  PROC Writelnt   (VAR r: Rider; x: INT );
  PROC WriteReal  (VAR r: Rider; x: REAL);
  PROC WriteString(VAR r: Rider; x: ARRAY OF CHAR);

  PROC Delete(   name: ARRAY OF CHAR; VAR res: INT);
  PROC Rename(old,new: ARRAY OF CHAR; VAR res: INT);
END Files.
\end{verbatim}
The procedures are listed in 4 groups:
\begin{itemize}
  \item[$1^{st}$] group operates on files,
  \item[$2^{nd},3^{rd}$] on riders for reading and writing, respectively, and
  \item[$4^{th}$] on the file directory only.
\end{itemize}
We assume that files are stored on a persistent medium, such as a magnetic disk or flash-rom,
and that all files are listed in a directory with names. In the
\begin{itemize}
  \item[$1^{st}$] group, procedure \verb|Old| yields the file listed in the directory by
    the specified name. \verb|New| generates a new (empty) file with specified name.
    Actual registration in the directory is performed by procedure \verb|Register|,
    typically after the entire file had been generated. \verb|Close| terminates the writing
    of a file (flushing buffers), and it is implied in \verb|Register|. \verb|GetDate|
    yields the date and time of a file’s creation.
  \item[$2^{nd}$] group, \verb|Set| places a rider on a file at a specified position
    (between $0$ and the length of the file). \verb|Pos| indicates the rider's current
    position, and \verb|Base| the file on which it is placed. The \verb|Read| procedures
    read a specified type of data element (consisting of one or several bytes) without any
    transformation of representation. They advance the rider by the appropriate amount.
    A string is assumed to be terminated by a \verb|0X| character.
  \item[$3^{rd}$] group, procedures operate similarly for generating (writing) a file.
    Note that the rider must not necessarily be positioned at the end of the file,
    although this is the normal, prevalent case.
\end{itemize}

We are now in a position to show, how files are typically written and read in Oberon. Let
us assume the declarations
\begin{verbatim}
  f: Files.File; r: Files.Rider
\end{verbatim}
\begin{itemize}
  \item[$1^{st}$,] the empty file is created,
  \item[then,] a rider is associated with it.
  \item[We assume] it is generated sequentially (here in a while-loop), and
  \item[finally] (and optionally) it is registered in the directory.
\end{itemize}
\begin{verbatim}
  f := Files.New(“MyFile”);
  Files.Set(r, f, 0);
  WHILE more DO compute(next);
                Files.Write(r, next) END;
  Files.Register(f)
\end{verbatim}
The file can thereafter be read by the following pattern:
\begin{itemize}
  \item[$1^{st}$,] the specified name file is associated with variable $f$.
  \item[Then] a rider $r$ is placed at the start of $f$, and
  \item[then] advanced by reading.
\end{itemize}
\begin{verbatim}
  f := Files.Old(“MyFile”); Files.Set(r, f, 0):
  WHILE ~r.eof DO Files.Read(f,next); process(next) END
\end{verbatim}
Note that the rider indicates that the end of the file had been reached after the 1st
unsuccessful attempt of reading.

\section{Texts, Readers and Writers}
\label{sec:texts}
\verb|Texts|, basically, are sequences of characters. In the Oberon System, however, texts
have some additional properties. As they can be displayed and printed, they must carry
properties which determine their style and appearance. In particular, a font is specified.
This is a style, the patterns by which characters are represented. Subsequences of charcaters
in a text may have their individual fonts. Furthermore, Oberon texts specify color and
vertical offset (allowing for negative offset for subscripting, and positive offset for
superscripting). Texts are typically subjected to complicated editing operations, which
require a flexible, internal data representation. Therefore \verb|Texts| in Oberon differ
substantiall from \verb|Files| in many respect. However, in their essence they are also
sequences, and the basic operations strongly resemble those of files.

Our special interest in texts is justified by the fact, that humans communicate with
computers mostly via texts, be they typed on a keyboard or displayed on a screen. As a
consequence, reading and writing of texts usually includes a change of data representation.
For example, integers will have to be changed into sequences of decimal digits on output,
and the reading of integers requires the reading of a sequence of digits and the computation
of the represented value. These conversions are included in the read/write procedures of
module \verb|Texts|, which we have used in many examples in preceding chapters. The
following is its definition, which evidently resembles that of \verb|Files|. In place of
the single type \verb|Rider| we find the 2 types \verb|Reader| and \verb|Writer|.
\begin{verbatim}
DEFINITION Texts;
  IMPORT Files, Fonts:
  CONST replace = 0; insert = 1; delete = 2;
        Inval = 0; Name = 1; String = 2; Int = 3;
        Real = 4; LongReal = 5; Char = 6;
  TYPE Text = POINTER TO RECORD len: LONGINT END;
       Reader = RECORD eot: BOOL; fnt: Fonts.Font;
                       col, voff: SHORTINT END;
       Scanner = RECORD (Reader) nextCh: CHAR;
                   line, class: INT: i: LONGINT;
                   x: REAL; y: LONGREAL;
                   c: CHAR: len: SHORTINT;
                   s: ARRAY 32 OF CHAR;
                 END; Buffer;
       Writer = RECORD buf: Buffer; fnt: Fonts.Font;
                       col, voff: SHORTINT END;

  PROC Open       (t: Text; name: ARRAY OF CHAR);
  PROC Delete     (t: Text; beg, end: LONGINT);
  PROC Insert     (t: Text; pos:LONGINT; b:Buffer);
  PROC Append     (t: Text; b: Buffer);
  PROC ChangeLooks(t: Text; beg, end: LONGINT;
                   sel: SET; fnt: Fonts.Font;
                   col, voff: SHORTINT);

  PROC OpenReader(VAR r: Reader; t: Text; pos: LONGINT);
  PROC Read      (VAR r: Reader; VAR c: CHAR);
  PROC Pos       (VAR r: Reader): LONGINT;

  PROC OpenScanner(VAR S: Scanner; t: Text; pos: LONGINT);
  PROC Scan       (VAR S: Scanner);

  PROC OpenWriter   (VAR w: Writer);
  PROC SetFont      (VAR w: Writer; fnt: Fonts.Font);
  PROC SetColor     (VAR w: Writer; col: SHORTINT);
  PROC SetOffset    (VAR w: Writer; off: SHORTINT);
  PROC Write        (VAR w: Writer; c: CHAR);
  PROC WriteLn      (VAR w: Writer);
  PROC WriteString  (VAR w: Writer; s: ARRAY OF CHAR);
  PROC Writelnt     (VAR w: Writer; x, n: LONGINT);
  PROC WriteHex     (VAR w: Writer; x: LONGINT);
  PROC WriteReal    (VAR w: Writer; x: REAL;     n: INT);
  PROC WriteRealFix (VAR w: Writer; x: REAL;  n, k: INT);
  PROC WriteLongReal(VAR w: Writer; x: LONGREAL; n: INT);
  PROC WriteDate    (VAR w: Writer; t, d: LONGINT);

  PROC Load (VAR r: Files.Rider; t: Text);
  PROC Store(VAR r: Files.Rider; t: Text)
END Texts.
\end{verbatim}
Procedure \verb|OpenWriter| uses as defaults a standard font, black color, and zero offset.

Simple, sequential reading and writing of a text now follow the patterns of reading and
writing a file.  Let us assume the following declarations of variables:
\begin{verbatim}
  T: Texts.Text;
  R: Texts.Reader;
  W: Texts.Writer;

  NEW(T); Texts.Open(T, “MyName”);
  Texts.OpenReader(R, T, 0); Texts.Read(R, ch);
  WHILE ~R.eot DO process(ch); Texts.Read(R, ch) END;

  NEW(T); Texts.Open(T, “MyName");
  Texts.OpenWriter(W); generate(ch);
  WHILE more DO Texts.Write(W, ch); generate(ch) END;
  Texts.Append(T, W.buf)
\end{verbatim}
Note that the Oberon paradigm is to write a text, or a piece of text, first into a buffer,
and thereafter insert or append it to a text. This is done for reasons of efficiency,
because the possibly needed rendering of the text, for example on a display, can be done
once upon insertion of the buffered piece of text rather than after generating each character.

Very often one wants to read a text consisting of a sequence of items which are not all of
the same type, such as numbers, strings, names, etc. When using procedures reading items
of a fixed type, the programmer must know the exact sequence in which the various items will
appear in the text. Typically one does not know, and even if a specific order is specified,
mistakes may occur. So one should like to have available a reading mechanism that reads
items one at a time, but lets the program determine what type of item was read, and take
further steps accordingly. This facility is provided in the Oberon text module by the machanism
called \emph{scanning}. In place of a \verb|Reader| we use a \verb|Scanner|.  Procedure
\verb|Scan(S)| then reads the next item. Its kind can be determined by inspecting the field
\verb|S.class|, and its value accordingly is given by \verb|S.l| in the case an integer was
read, \verb|S.x| if a real number was read, \verb|S.s| if a name or a string was read. This
scheme has proven to be most useful because of its flexibility. It defines the following
syntax for texts to be scanned. Blanks and line ends serve to separate consecutive items.
\begin{verbatim}
  item = name | integer | real | longreal | string
              | SpecialChar
  name = letter {letter | digit}
  sign = [-]
  decimal = [sign] digit {digit}
  hexdig = A | B | C | D | E | F
  integer = decimal | digit {digit | hexdig} H
  real = decimal . digit {digit} [(E|D) decimal]
\end{verbatim}
A string is any sequence of any characters (except quotes) enclosed in quotes. Special
characters are all characters except letters, digits, and the quote mark.

\section{Standard Input and Output (IO)}
\label{sec:stdio}
In order to simplify the description of IO of texts in simple cases, the Oberon system
introduces some conventions and global variables. We call this source of input and sink
of output \emph{standard IO}.

Standard output sink is the text \verb|Log| defined as global variable in module Oberon.
Given a global writer \verb|W|, the text, previously written by \verb|Write| procedures into
the writers buffer (\verb|W.buf|) is sent to the displayed \verb|Log| text by the statement
\begin{verbatim}
  Texts.Append(Oberon.Log, W.buf)
\end{verbatim}
The standard input is often assumed to be the text following the command which invoked
execution of a given procedure. This input text is identified by the variable \verb|Par|
in module \verb|Oberon|. It is considered as parameter of the invoked command, and it is
typically read by the scanning mechanism described above. The necessary statements are:
\begin{verbatim}
  Texts.OpenScanner(S, Oberon.Par.text,
                       Oberon.Par.pos);
  Texts.Scan(S)
\end{verbatim}
The 1st item thus read may be the 1st item of the desired text, or it may be the name of
the text file to be scanned, or something else, according to the specification of the
individual command. The following conventions have established themselves for the
designation of input texts. Assume that the command (procedure) name is followed by:
\begin{itemize}
  \item an identifier (possibly qualified). Then this is the name of the input text (input file),
  \item an asterisk (*). Then the marked viewer (window) contains the input text,
  \item an @ symbol. Then the most recent text selection is the beginning of the input text,
  \item an arrow (↑). Then the most recent selection denotes the file name of the input text.
\end{itemize}
These conventions are expressed by the following function yielding the designated input text:
\begin{verbatim}
  PROC This*(): Texts.Text;
    VAR beg, end, time: LONGINT; S: Texts.Scanner;
        T: Texts.Text: v: Viewers.Viewer;
  BEGIN Texts.OpenScanner(S, Oberon.Par.text,
                             Oberon.Par.pos);
    Texts.Scan(S);
    IF S.class = Texts.Char THEN
      IF S.c="*" THEN (*input in marked viewer*)
        v := Oberon.MarkedViewer();
        IF (v.dsc # NIL) &
           (v.dsc.next IS TextFrames.Frame) THEN
          T := v.dsc.next(TextFrames.Frame).text;
          beg := 0
        END
      ELSIF S.c="@" THEN (*input starts at selection*)
        Oberon.GetSelection(T, beg, end, time);
        IF time < 0 THEN T := NIL END
      ELSIF S.c = “↑” THEN (*selection is the file name*)
        Oberon.GetSelection(T, beg, end, time);
        IF time >= 0 THEN (*there is a selection*)
          Texts.OpenScanner(S, Oberon.Par.text,
                               Oberon.Par.pos);
          Texts.Scan(S);   (*input is named file*)
          IF S.class = Texts.Name THEN
            NEW(T); Texts.Open(T, S.s); pos := 0
          END
        END
      END
    ELSIF S.class = Texts.Name THEN (*input is named file*)
      NEW(T); Texts.Open(T, S.s); pos := 0
    END;
    RETURN T
  END This;
\end{verbatim}
Let us assume that this function procedure is defined in module \verb|Oberon|, together
with the global variable \verb|pos|, indicating the position in the text where input is
to start. The following program for copying a text may serve as a pattern for selecting
the input according to the conventions postulated above.
\begin{verbatim}
  MODULE ProgramPattern;
    IMPORT Texts, Oberon:
    VAR W: Texts.Writer; (*global writer*)
 
    PROC copy(VAR R: Texts.Reader):
      VAR ch: CHAR;
    BEGIN Texts.Read(R, ch);
      WHILE ~R.eot DO Texts.Write(W, ch);
                      Texts.Read(R, ch) END
    END copy
 
    PROC SomeCommand*:
      VAR R: Texts.Reader; (*local reader*)
    BEGIN Texts.OpenReader(R, Oberon.This.text,
                              Oberon.Par.pos);
      copy(R, W); Texts.Append(Oberon.Log, W.buf)
    END SomeCommand;
  BEGIN Texts.OpenWriter(W)
  END ProgramPattern.
\end{verbatim}
Module \verb|Oberon| can be considered as the core of the Oberon system housing a small
number of global resources. These include the previously encountered output \verb|Log|,
and the record \verb|Par|, specifying the viewer, frame, and text in which the activated
command lies, and hence providing access to its input parameters. Here we show only an
excerpt of the definition of module \verb|Oberon|:
\begin{verbatim}
  TYPE ParList = POINTER TO RECORD
                   vwr: Viewers. Viewer;
                   frame: Display.Frame;
                   text: Texts. Text:
                   pos: LONGINT;
                 END;
  VAR Log: Texts.Text: Par: ParList;
  PROC Time(): LONGINT;

  (*provide coordinates X, Y
                for a new viewer to be allocated*)
  PROC AllocateUserViewer  (DX: INT; VAR X,Y: INT);
  PROC AllocateSystemViewer(DX: INT; VAR X,Y: INT);
 
  PROC GetSelection(VAR           text: Texts.Text;
                    VAR beg, end, time: LONGINT);
\end{verbatim}
To conclude this introduction to Oberon conventions about IO, we show how a new viewer
(window) is opened, and how the output text is directed into this text viewer. Optimal
positioning of the new viewer is achieved by procedure \verb|Oberon.AllocateViewer|,
with the specified set of standard, frequently needed commands in the title bar.
\begin{verbatim}
  PROC OpenViewer(T: Texts.Text);
    VAR V: MenuViewers.Viewer; X, Y: INT;
  BEGIN
    Oberon.AllocateUserViewer(Oberon.Mouse.X, X, Y);
    V := MenuViewers.New(TextFrames.NewMenu("Name",
          "Sys.Close .Copy .Grow Ed.Search .Store"),
                         TextFrames.NewText(T, 0),
                         TextFrames.menuH, X, Y
    )
  END OpenViewer;
\end{verbatim}
