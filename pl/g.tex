\chapter{Record Types}
In an array all elements are of the same type. In contrast to the array, the record structure offers the
possibility to declare a collection of elements as a unit even if the elements are of different types.
The following examples are typical cases where the record is the appropriate choice of structuring
method. A date consists of three elements, namely day, month, and year. A description of a person
may consist of the person's names, sex, identification number, and birthdate. This is expressed by
the following type declarations
\begin{verbatim}
  Date = RECORD day,month,year: INT END
  Person = RECORD
    firstName,lastName: ARRAY 32 OF CHAR;
    male: BOOL;
    idno: INT;
    birth: Date
  END
\end{verbatim}
The record structure makes it possible to refer either to the entire collection of data or to individual
elements. Elements of a record are also called record fields, and their names are called field
identifiers. This stems from the habit of looking at such data as forms or tables drawn on paper with
individual fields delineated as rectangles and labelled with field names. Similar to arrays, where we
denote the $i$-th element of an array $a$ by $a[i]$, i.e. by the array identifier followed by an index,
we denote the field $f$ of a record $r$ by $r.f$, i.e. by the record identifier followed by the field's
name. For example, given the variables
\begin{verbatim}
  d1, d2: Date;
  p1, p2: Person;
  students: ARRAY 100 OF Person
\end{verbatim}
we can construct, for example, the following variable designators:
\begin{verbatim}
  d1.day
  d2.month
  p1.firstName
  p1.birth
\end{verbatim}
These examples show that fields may themselves be structured. Similarly, records may be
elements of array or record structures, i.e. there exists the possibility to construct hierarchies of
structures. As a consequence, selectors of elements can be sequenced, as shown by the following
examples of designators. The multi-dimensional array discussed in the chapter on arrays now
appears as a particular case of these structuring hierarchies.
\begin{verbatim}
  p1.lastName[0]
  p2.birth.year
  students[23].idno
  students[k].firstName[0]
\end{verbatim}
At first sight the record may appear as a generalized array, because it relaxes the restriction that all
elements be of the same type. However, in another aspect it is more restrictive than the array: the
selector of the element must be a fixed field identifier, whereas the index selecting an array element
may be an expression, i.e. a result of previous computations.

It is important to note that a record may assume arbitrary combinations of its field's values. Hence,
in the example of the type Date, a value day = 31 may coexist with month = 2, although this is not
an actual date.

The syntax of a record declaration is defined as follows.
\begin{verbatim}
  RecordType = RECORD FieldListSequence END.
  FieldListSequence = FieldList {; FieldList}.
  FieldList = [IdentList: type].
\end{verbatim}
and that of a designator is
\begin{verbatim}
  designator = qualident {selector}.
  selector = "." identifier | "[" ExpList "]" | "^".
\end{verbatim}
\textbf{Note}: Oberon’s concept of record type extension is not discussed in this book.
