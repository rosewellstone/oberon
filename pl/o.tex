%\chapter{The Syntax of Oberon}
\chapter{Library}
\section{Symbols and Keywords}
\begin{verbatim}
  (  )   + - * /   <  = >    . , ;  :  :=  ..
  [  ]   & | ~ ^   <= # >=   ARRAY    IMPORT
  {  }     DIV      CASE     BEGIN    MODULE
           END      ELSE     CONST    RECORD
  BY IS    FOR      PROC     ELSIF    REPEAT
  DO OF    MOD      THEN     FALSE    RETURN
  IF OR    NIL      TRUE     UNTIL
  IN TO    VAR      TYPE     WHILE    POINTER
\end{verbatim}

\section{Standard Functions}
\begin{table}[h!]
  \centering
  \begin{tabular}{l c c l}
    Name     & ArgType & ResType& Function \\\hline
    ABS(x)   & numeric & type(x)& absolute value \\
    ODD(x)   & INT     & BOOL& x MOD 2 = 1 \\
    LEN(v)   & array   & INT & length \\
    LSL(x,n) & INT     & INT & logical shift left \\
    ASR(x,n) & INT     & INT & signed shift right \\
    ROR(x,n) & INT     & INT & rotated right \\
    ORD(x)   & CHAR|   & INT & ordinal \# of \\
             & BOOL|SET \\
    CHR(x)   & INT     & CHAR& character of \\
    FLOOR(x) & REAL    & INT & round down \\
    FLT(x)   & INT     & REAL& identity \\
  \end{tabular}
\end{table}

\section{Standard Procedures}
\begin{table}[h!]
  \centering
  \begin{tabular}{l c l}
    Name      & ArgType  & Procedure \\\hline
    INC(v)    & INT      & v := v + 1 \\
    INC(v, n) & INT      & v := v + n \\
    DEC(v)    & INT      & v := v - 1 \\
    DEC(v, n) & INT      & v := v - n \\
    INCL(v, x)& SET, INT & v := v + {x} \\
    EXCL(v, x)& SET, INT & v := v - {x} \\
    ASSERT(b) & BOOL     & abort if \~{}b \\
    NEW(v)    & pointer type & allocate v\^{}
  \end{tabular}
\end{table}
