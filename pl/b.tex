\chapter{The Concept of Locality}
If we inspect the preceding example of procedure \verb|Add|, we notice that the role of the
variable $i$ is strictly confined to the procedure body. This inherent locality should be expressed
explicitly, and can be done by declaring $i$ inside the procedure declaration. $i$ thereby becomes
a local variable.
\begin{verbatim}
  PROC Add;
    VAR i: INT;
  BEGIN sum := 0.0;
    FOR i:=0 TO N-1 DO
      sum := a[i] + sum
    END
  END Add
\end{verbatim}
In some sense, the procedure declaration assumes the form of a separate program. In fact, any
declaration possible in a program, such as constant, type, variable, or procedure declaration,
may also occur in a procedure declaration. This implies that procedure declarations may be
nested and are defined recursively.

It is good programming practice to declare objects local, i.e. to confine the existence of an object
to that procedure in which it has meaning. The procedure, i.e. the section of program text in which
a name is declared, is called its scope. Since declarations can be nested, scopes are also nestable.
The possibility of having objects local to some scope has several consequences. For example, the
same name can be used to denote different objects. In fact, this is a most useful consequence,
because a programmer is free to choose local identifiers without knowledge of those existing in the
surrounding scope, as long as they do not denote objects used in the local scope (in which case he
must obviously be aware of them anyhow). This decoupling of knowledge about different program
parts is particularly useful and perhaps even vital in the case of large programs.

The rules of scope (validity of identifiers) are as follows:
\begin{enumerate}
  \item The scope of an identifier is the procedure in which its declaration occurs, and all
    procedures enclosed by that procedure, subject to rule 2.
  \item If an identifier $i$ declared in a procedure $P$ is redeclared in some inner procedure $Q$
    enclosed in $P$, then procedure $Q$ and all procedures enclosed in $Q$ are excluded from the
    scope of $i$ declared in $P$.
  \item The standard identifiers of Oberon are considered to be declared in an imaginary scope
    enclosing the program.
\end{enumerate}
These rules may also be remembered by the algorithm in which the declaration of a given identifier
$i$ is searched:
\begin{itemize}
  \item[] First, search the declarations of the procedure $P$ in whose body $i$ occurs; if the
    declaration of $i$ is not among them, continue the search in the procedure surrounding $P$;
  \item[] Then repeat this same rule until the declaration is encountered.
\end{itemize}
\begin{verbatim}
  VAR a: INT;
  PROC P;
    VAR b: INT;
    PROC Q;
      VAR b, c: BOOL;
    BEGIN
      (* here a, b(BOOL), c are visible *)
    END Q;
  BEGIN
    (* here a, b(INT) are visible *)
  END P
\end{verbatim}

A consequence of the locality concept and of the rule that a variable does not exist outside its
scope is that its value is lost when its declaring procedure is terminated. This implies that, when the
same procedure is later called again, this value is unknown. The values of local variables are
undefined when the procedure is (re)entered. Hence, if a variable must retain its value between two
calls, it must be declared outside the procedure. The "lifetime" of a variable is the time during which
its declaring procedure is active.

The use of local declarations has 3 significant advantages:
\begin{enumerate}
  \item It makes clear that an object is confined to a procedure, usually a small part of the entire
    program.
  \item It ensures that inadvertent use of a local object by other parts of the program is detected
    by the compiler.
  \item It enables the implementation to minimize storage, because a variable's storage is released
    when the procedure to which the variable is local is terminated. This storage can then be reused
    for other variables.
\end{enumerate}
