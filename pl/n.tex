\chapter{Object-oriented Programming (OOP)}
\section{Origins of OOP}
The term OOP originated around 1975. It expressed a shift in the paradigm of programming.
\begin{itemize}
  \item In the conventional, \emph{procedural} programming style, the algorithm stands in
    the foreground of the programmer's concerns, the algorithm operating on a set of data.
  \item In the OO view, the datum on which algorithms are applied stand in the foreground,
    They are grouped into what is called an \emph{object}.
\end{itemize}

Programming is a notoriously difficult task, and the rapidly increasing power of modern computers
made the situation worse. The tasks to be modeled and solved became more and more complicated.
Their complexity often surpassed the capabilities of the human mind. In this situation, it is not
surprising that the demand for better tools, for a new discipline, even for panaceas became urgent
and ardent. The new paradigm of OOP was therefore welcomed, as their proponents raised hopes for
an easier approach to complex problems. Partly, these hopes were exaggerated, partly the essence
of OOP was driven to unwise extremes for the sake of unity of doctrine. For example, even simple
integers were to be regarded as objects in the new sense. What was this “new sense”?

Before we try to answer this question, we should point out that the notion, but not the term OOP
was introduced at least 10 years earlier. In 1965 the language Simula was introduced as a variant
of Algol adapted to and extended for the needs of simulation of systems governed by discrete
events. (What later became objects were then called elements). A well-known example of such a
system was a supermarket. Here customers, merchandise, cashiers, personnel, shelves, carts, were
all modeled as objects, and they formed classes. For modeling such systems, this was the obvious
approach. Only 10 years later, under the leadership of Smalltalk, was the new viewpoint extended
to other applications. Of foremost interest were operating systems (OS), where resource pools,
processors, windows, texts and devices became regarded as autonomous agents. Graphics editors
were another suitable application. They are used to handle various figures such as lines, circles,
rectangles, ellipses, text captions, each forming their own class.

In these examples, the objects constitute almost autonomous entities, in the simulated models as
well as the operating systems. They not only feature attributes and properties common to all
members of their class, but also a common behavior represented by associated procedures. These
are typically activated by a “higher power” such as a scheduler in the case of simulation, or a
human being in the case of an OS or a graphics editor. Both data and procedures together
constitute and characterize objects.

The Smalltalk designers clearly wished to present not only a new language, but also a new paradigm,
a new approach to programming. To be effective and convincing in this endeavor, they provided also
a new terminology to underscore a different quality of programming. In this effort,
record-structured variables became objects, their associated procedures became methods, a data type
became a class, and procedure calls are now termed sending messages to objects. This is denoted as
\begin{verbatim}
  object.method (message)
\end{verbatim}
In Simula, the ancestor of OOP, there was not a set of methods to be invoked by messages, but
rather a single coroutine representing the algorithmic behavior of the simulated object, which was
regarded as continuously alive. We may now consider the parts between breakpoints as methods,
although the analogy is somewhat lacking. For example, a customer of the supermarket would be
characterized by 1st entering the scene, picking up a cart, wandering along the shelves, picking
up merchandise, passing a cashier, and finally returning the cart and leaving the scene. Objects
(customers) would emerge (be allocated), proceed, and leave (be deallocated). Hence, typically a
dynamic data structure, a linked list, is used to represent the set of currently involved objects.

It now seems that OOP merely provides (or dictates?) a different view of familiar programming
concepts, but does not require any additional features in a language, except perhaps some notational
conventions. The same old contents in a new form? Subsequently, we will investigate this question.

\section{Type extensions and inhomogeneous data structures}
Let us consider a simple, rudimentary line drawing editor. It is typical for many applications in so
far as it operates on a set of objects. In this case, the objects are straight lines, rectangles,
circles, ellipses, text captions, and others. These objects are typically represented as a linked
list, to which it is easy to add and delete new instances.

However, using a statically typed language, we immediately encounter a difficulty. Declaring each of
the figures by its own record type, we are prevented from forming a single list. Declaring a separate
list for each type of figure appears as artificial and cumbersome. On the other hand, integrating all
figure types into a single declaration appears as equally contorted and cumbersome:
\begin{enumerate}
  \item a discrimination field is necessary, subsequently giving rise to various IF statements.
  \item a single declaration will contain record fields that pertain to only a few, but not to all
    figure types.
\end{enumerate}
Whatever one chooses to do, the solution remains unsatisfactory.

We therefore introduce a new feature in the language Oberon. It is called type extension. Basically,
in our example, we declare a base type called \verb|Figure|, and then a derived type for each
individual kind of figure, here called \verb|Line|, \verb|Rectangle|, \verb|Circle|, etc. The key
is that each instance of these subtypes is also considered as of its base type \verb|Figure|. Such
declarations of derived types have the form:
\begin{verbatim}
  RecordType = RECORD (Base) FieldListSequence END
\end{verbatim}
The \verb|Base| specifies the base type of the new, derived record type, which thereby becomes a
subtype. Evidently, it now becomes possible to define entire types hierarchies. For our example:
\begin{verbatim}
  TYPE Figure = POINTER TO FigureDesc;
  FigureDesc  = RECORD (*this is the base type*)
    x,y,w,h: INT; (*coordinates, width & height*)
       next: Figure (*next in list of figures*)
  END;
  LineDesc      = RECORD (FigureDesc) END;
  RectangleDesc = RECORD (FigureDesc) lw : INT END;
  CircleDesc    = RECORD (FigureDesc) r  : INT END;
  EllipseDesc   = RECORD (FigureDesc) a,b: INT END;
  CaptionDesc   = RECORD (FigureDesc) cap:
                               ARRAY 32 OF INT END
\end{verbatim}
Accordingly, a descriptor of a rectangle consists of the fields $x$, $y$, $w$, $h$, and $lw$ (line width),
the one for a circle of the fields $x$, $y$, $w$, $h$, $r$(adius), the one for an ellipse of $x$, $y$,
$w$, $h$, $a$, $b$, and the one for a text caption of $x$, $y$, $w$, $h$, $cap$. The derived types retain
all fields of the base type. In OO terminology, they are said to inherit the properties of the base. We
will not adopt this anthropomorphic term.

We call the derived types extended types, because they extend the base type with additional properties.
In OO terminology, the base type is called an abstract type, on which the concrete types, the extended
types, rest. All of them are compatible also with their base type.  Therefore, it is possible to link
them into a single list through the base type’s field next, and therefore to build data structures that
contain elements of different types, that is, inhomogeneous structures.  For many applications, this is
an important and necessary requirement.

When, for example, traversing such an inhomogeneous list, it may be necessary to determine the type of
each encountered object. It is only known that every element in the list is of type \verb|Figure|, but
now we wish to determine the subtype of an individual element. This is possible through the new facility
of the type test, which is classified as a \verb|Boolean| expression, and has the form (syntax)
\begin{verbatim}
  expression = variable IS identifier
\end{verbatim}
Assume a variable $p$ of type \verb|Figure|, then the traversal of a list and the discriminated application
of a drawing procedure can be expressed as follows:
\begin{verbatim}
  WHILE p # NIL DO
       IF p IS Line   THEN DrawLine(p)
    ELSIF p IS Circle THEN DrawCircle(p)
    END ;
    p := p.next
  END
\end{verbatim}
We will discover in the next section that there is actually a better, shorter way of expressing this
action.

The type test is a necessary feature, because Oberon has deviated from the dogma of strictly static
data typing. Note, however, that the actual type \verb|T| of a variable declared to be of base type
\verb|T0| can only be a subtype of \verb|T0|. A \verb|Figure| can be a \verb|Line|, \verb|Circle|,
etc., but not, for example, an integer. Type tests are therefore not needed for most data accesses.
Furthermore, type tests can be implemented very efficiently.

In addition to the type test, Oberon also offers the construct of type guard, which has the form of
a selector in variable designators.
\begin{verbatim}
  designator = qualident {. ident | "[" ExpList "]"
                                  | (qualident) | ^ }
\end{verbatim}
Consider the example of the designator
\begin{verbatim}
  p(Circle).r
\end{verbatim}
The simple designator \verb|p.r| would not be acceptable, because $p$ is of type \verb|Figure|,
which does not feature a field $r$. With the type guard, the programmer can ascertain that in this
case $p$ is also of type \verb|Circle|, in which case the field $r$ is indeed applicable. Whereas
$p$ is of base type \verb|Figure|, \verb|p(Circle)| is of type \verb|Circle|. Of course, it must be
assumed that the programmer has made sure that his claim is correct, and the system will have to
check whether this is indeed so (at run time). The type guard in a sense resembles an array index,
where a system must check whether the actual index lies within the specified array bounds.

\section{Methods}
It has already been remarked that objects are characterized not only by their attributes (data),
but also by their behavior, i.e. procedures. Just as attributes may differ among various subtypes,
so may their behavior. Therefore, it is sensible to associate the procedures directly with individual
objects or at least their subtypes. In strongly OO languages this is achieved by letting class
declarations specify associated procedure declarations. In Oberon, we use what is available and
refrain from introducing new features. We simply add to a record’s data fields other fields that
represent procedures, that is, are of procedure types. We redefine our example of the type
\verb|Figure| accordingly. Let us assume the 2 operations of drawing and marking figures.
\begin{verbatim}
  FigureDesc = RECORD
                 x, y, w, h: INT;
                 next: Figure;
                 draw: PROC(f: Figure);
                 mark: PROC(f: Figure; on: BOOL)
               END;
\end{verbatim}
Whereas in strictly OO languages methods are automatically associated with every object upon its
creation, in Oberon this must be achieved by explicit assignments. Such initial assignments are
called installations (of procedures). For example:
\begin{verbatim}
  VAR c: Circle; NEW(c); c.x := ...;
  c.draw := DrawCircle;
  c.mark := MarkCircle; ...
\end{verbatim}
The call of a method, i.e. of an object-bound procedure, is expressed as in the following example:
\begin{verbatim}
  c.mark(c, TRUE)
\end{verbatim}
The association of type-specific procedures to every object has the substantial advantage that
subclass-discriminating conditional statements can be avoided. For example, the above call
automatically activates the appropriate \verb|MarkCircle| procedure without having to execute a
sequence of \verb|if| statements distinguishing between lines, circles, rectangles etc. This
contributes to efficiency. The drawing of all objects in a list is now expressed quite simply as
\begin{verbatim}
  WHILE p # NIL DO p.draw(p); p := p.next END
\end{verbatim}
and in each call automatically the appropriate drawing procedure is activated.

\section{Handlers and Messages}
In Part III of this text, the importance of proper decomposition (modularization) of systems was
emphasized. A module restricts the visibility of declarations and incorporates the concept of
information hiding. In OO languages a type definition, i.e. a class declaration, assumes the role of a
module, also in the sense of unit of separate compilation. In Oberon, we prefer to keep the constructs
expressing information hiding, and those for type and procedure definition separated and independent.
Hence, every module may contain one or several type definitions, and perhaps none at all.

Nevertheless, considering a data type and its operators as a logical unit, and to hide details of
implementation, are old and proven techniques manifest in the concept of the abstract data type.
Particularly for complex objects it is highly desirable that the various subtypes can be defined and
implemented by separate teams of programmers, and this perhaps years after the definition of the basis
was completed and had been published or distributed. It is desirable to be able to define every derived
type in its own module. This demand also holds for base types, with operators applying to the set of
objects (in contrast to individual objects) defined in the base module, as well as for those defining
the derived type in client modules. Among such operations we mention the so-called broadcast, the
traversal of the objects’ data structure (list, tree) with application of a method to every element.
Declarations of such broadcasts are thus inherently confined to the base module.

Now it may happen that a subtype introduces its specific operations, including some that are not
shared by other subtypes. Obviously, it is desirable that also these operations can be broadcast with
the respective procedure defined in the base module, even if the new operators were introduced years
after the base had been established. The base, however, cannot be changed nor accessed, because in the
meantime it might have been distributed to many customers. How can this dilemma be resolved?

The solution lies in replacing the set of methods by a single procedure, which discriminates between
the various methods. Such a single procedure we call a handler. It has only 2 parameters: The object
to which it is applied, and an identification of the operation with its actual parameters. The 2nd
parameter has the form of a record, and we call it a message. For such a message type to be capable of
denoting any operation, even those to be introduced in the future, the same feature is used as for
adding new types derived from old ones: \emph{Type Extension}. Let us explain this solution by our
example of the graphics editor.

First, the declaration of the base type now becomes
\begin{verbatim}
  FigureDesc = RECORD
                 x, y, w, h: INT;
                       next: Figure;
                     handle: Handler
               END;
\end{verbatim}
with the procedure type \verb|Handler| being defined as
\begin{verbatim}
  Handler = PROC(f: Figure; VAR msg: Message)
\end{verbatim}
and the new type \verb|Message| by the (extensible) null record
\begin{verbatim}
  Message = RECORD END
\end{verbatim}

If certain operations applying to all objects are already known at the time of defining the base type
(which is usually the case), the respective message (sub)types are directly declared, as for example:
\begin{verbatim}
  DrawMsg = RECORD (Message) END;
  MarkMsg = RECORD (Message) on: BOOL END;
  MoveMsg = RECORD (Message) dx, dy: INT END
\end{verbatim}
As an aside, we note that the record field next need not be exported from the base module, thus
keeping the structure of the set of objects (list, tree) and its operations, such as traversal, hidden
and flexible. A broadcast can now be specified as follows:
\begin{verbatim}
  PROC Broadcast(VAR msg: Message);
    VAR f: Figure;
  BEGIN f := root; (*global variable in base module*)
    WHILE f # NIL DO
      f.handle(f, msg); f := f.next
    END
  END Broadcast
\end{verbatim}
Whenever a concrete figure type is introduced, this will typically be done in a new client module
importing the base module \verb|Figure|. In addition to the new subtype, for example \verb|Rectangle|,
its associated handler is declared following the pattern shown below. This handler is installed in the
field handle of every object of type \verb|Rectangle|.
\begin{verbatim}
  PROC Handle(f: Figure; VAR msg: Message);
    VAR r: Rectangle;
  BEGIN r := f(Rectangle);
       IF msg IS DrawMsg THEN (*draw rectangle r*)
    ELSIF msg IS MarkMsg THEN
      MarkRectangle(r, msg(MarkMsg).on)
    ELSIF msg IS MoveMsg THEN
      INC(r.x, msg(MoveMsg).x);
      INC(r.y, msg(MoveMsg).y)
    ELSIF ...
    END
  END Handle
\end{verbatim}
The statement for moving a single object $f$ by displacements $dX$ and $dY$ is now somewhat complicated:
\begin{verbatim}
  VAR msg: MoveMsg; msg.dx := dX;
  msg.dy := dY;  f.handle(f, msg)
\end{verbatim}
However, we must keep in mind that mostly messages are sent to objects indirectly, that is, when the
recipient is not known by the sender. A good example is the broadcast:
\begin{verbatim}
  msg.dx := dX; msg.dy := dY; Broadcast(msg)
\end{verbatim}
The particular flexibility of this technique using handlers and messages - sometimes identified as
polymorphism - as well as its extensibility stems from the fact that “unknown” messages are simply
ignored. They “fall through” the cascade of if - elsif statements in the handlers of objects to which
the message does not apply.

Let us, for example, add a new module, say \verb|Blinker|, to our editor system. It contains the declaration
of the derived type \verb|Blinker| (\verb|BlinkerDesc|). Let it require a new method, causing the object to
blink in a certain rhythm. We represent it by the (derived) message \verb|BlinkMsg|. Then the call of the
handler of any other object in the global object list, through which a broadcast will propagate, will have
no effect, because the case
\begin{verbatim}
  ELSIF msg IS BlinkMsg THEN ...
\end{verbatim}
does not exist in the handler. The technique using handlers guarantees optimal, unlimited extensibility of
existing systems by merely adding new modules on top of an existing hierarchy.

We summarise the technique as follows:
\begin{enumerate}
  \item An object is represented by a pointer to a record. The record features a single, procedural field
    called \verb|handle|. The procedure assigned to \verb|handle| is called a \emph{handler}. It features
    2 parameters:
    \begin{itemize}
      \item the 1st designates the object to which the handler is applied;
      \item the 2nd identifies the operation to be selected. It is a \verb|VAR| parameter with record
        structure, and is called a \emph{message}.
    \end{itemize}
  \item The handler defines the behavior of an object. It is called with a message which is typically of
    an extension of type \verb|Message| specific to the object type.
  \item The message parameter specifies the action to be taken by its (sub)type. Its record fields
    constitute the parameters of the action. Their number and types are specific to the particular message
    type and action.
  \item The handler consists of a single if - elsif cascade. Type tests discriminate between the various
    message subtypes and thereby actions.
  \item Assigning the handler to an object is called \emph{installation}.
  \item An action is initiated by first setting up the message and then sending one to the object.
  \item Messages may be broadcast to all objects of a heterogeneous data structure without knowing its
    nature. Inapplicable messages will simply be ignored.
\end{enumerate}

This concludes our brief introduction to the OOP paradigm. We realize that almost no language features had to
be added to Oberon to support it. Apart from the already present facilities of records and of procedural types,
only the notion of type extension is both necessary and crucial. It allows to construct hierarchies of types
and to build inhomogeneous data structures. As a consequence of abandoning the rule of strictly static typing,
the introduction of dynamic type tests became necessary. The further facility of the type guard is merely one
of convenience.

The procedural style of programming, which is most appropriate in many applications, and the OO style can
co-exist in Oberon quite happily. If one chooses to program in the conventional, procedural style, one can
ignore the OO facilities, and they do not get in one’s way.
