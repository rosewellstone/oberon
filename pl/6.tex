\chapter{Control Structures}
It is a prime characteristic of computers that individual actions can be selected, repeated, or
performed conditionally depending on some previously computed results. Hence the sequence of
actions performed is not always identical with the sequence of their corresponding statements. The
sequence of actions is determined by control structures indicating repetition, selection, or
conditional execution of given statements.

\section{Repetitive Statements}
The most common situation is the repetition of a statement or statement sequence under control of
a condition: the repetition continues as long as the condition is met. This is expressed by the 
\emph{while statement}. Its syntax is
\begin{verbatim}
  WhileStatement = WHILE expr DO StatementSequence END
\end{verbatim}
and its corresponding action is:
\begin{enumerate}
  \item Evaluate the condition which takes the form of an expression yielding
    the value \verb|TRUE| or \verb|FALSE|;
  \item If the value is \verb|TRUE|, execute the statement sequence and then repeat with step 1;
    if the value is \verb|FALSE|, terminate.
\end{enumerate}
The expression is of type \verb|BOOL|. This will be further discussed in the chapter on data types.
Here it suffices to know that a simple comparison is a Boolean expression. An example was given
in the introductory example, where repetition terminates when the 2 comparands have become
equal. Further examples involving while statements are:
\begin{enumerate}
  \item Initially, let $q = 0$ and $r = x$; then count the number of times $y$ can be subtracted
    from $x$, i.e. compute the quotient $q = x$\verb| DIV |$y$, and remainder $r = x$\verb| MOD |$y$,
    if $x$ and $y$ are natural numbers.
    \begin{verbatim}
      WHILE r >= y DO r := r-y; q := q+1 END
    \end{verbatim}
  \item Initially, let $z = 1$ and $i = k$; then multiply $z$ $k$ times by $x$, i.e. compute
    $z = xk$, if z and k are natural numbers:
    \begin{verbatim}
      WHILE i > 0 DO z := z*x; i := i-1 END
    \end{verbatim}
\end{enumerate}
When dealing with repetitions, it is important to remember the following points:
\begin{enumerate}
  \item During each repetition, progress must be made towards meeting the goal, namely "getting
    closer" to satisfying the termination condition. An obvious corollary is that the condition
    must be somehow affected from within the repeated computation. The following statements are
    either incorrect or dependent on some critical precondition as stated.
    \begin{verbatim}
      WHILE i > 0 DO
        k := 2*k (*i is not changed*)
      END
 
      WHILE i # 0 DO
        i := 1-2 (*i must be even and positive*)
      END
 
      WHILE n # i DO
        n := n*i; i := i+1
      END
    \end{verbatim}
  \item If the condition is not satisfied initially, the statement is vacuous, i.e. no action
    is performed.

  \item In order to obtain a grasp of the effect of the repetition, we need to establish a
    relationship that is stable, called an invariant. In the division example above, this is the
    equation $q*y + r = x$ holding each time the repetition is started. In the exponentiation
    example it is $z * x^i = x^k$ which, together with the termination condition $i = 0$ yields
    the desired result $z = x^k$.

  \item The repetition of identical computations should be avoided (although a computer has
    infinite patience and will not complain). A simple rule is to avoid expressions within
    repetitive statements, in which no variable changes its value. For example, the statement
    \begin{verbatim}
      WHILE i < 3*N DO tab[i] := x + y*z + z*i;
                       i := i+1 END
    \end{verbatim}
    should be formulated more effectively as
    \begin{verbatim}
      n := 3*N; u := x + y*z;
      WHILE i < n DO tab[i] := u + z*i; i := i+1 END
    \end{verbatim}
\end{enumerate}

In addition to the while statement, there is the \emph{repeat statement}, syntactically defined as
\begin{verbatim}
  RepeatStatement = REPEAT StatementSequence UNTIL expr
\end{verbatim}
The essential difference to the while statement is that the termination condition is checked each
time after (instead of before) execution of the statement sequence. As a result, the sequence is
executed at least once. An advantage is that the condition may involve variables that are undefined
when the repetition is started.
\begin{verbatim}
  REPEAT i := i+5; j := j+7; k := i DIV j UNTIL k > 23
  REPEAT r := r-y; q := q+1 UNTIL r < y
\end{verbatim}

\section{Conditional Statements}
The conditional statement, also called if statement, has the form
\begin{verbatim}
IfStatement = IF expr THEN StatementSequence
          {ELSIF expr THEN StatementSequence}
                     [ELSE StatementSequence]
              END
\end{verbatim}
The following example illustrates its general form.
\begin{verbatim}
     IF R1 THEN S1
  ELSIF R2 THEN S2
  ELSIF R3 THEN S3
           ELSE S4 END
\end{verbatim}
The meaning is obvious from the wording. However, it must be remembered that the expressions
\verb|R1 ... R3| are evaluated one after the other, and that as soon as one yields the value
\verb|TRUE|, its corresponding statement sequence is executed, whereafter the \verb|IF| statement
is considered as completed. No further conditions are tested. Examples are:
\begin{verbatim}
     IF x = 0 THEN s :=  0
  ELSIF x < 0 THEN s := -1
              ELSE s :=  1 END

  IF ODD(k) THEN z := z*x END
  IF k > 10 THEN k := k-10; d := 1
            ELSE d := 0 END
\end{verbatim}

The constructs discussed so far enable us to develop a few simple, complete programs as
subsequently described. The 1st example is an extension of our introductory example computing
the greatest common divisor $gcd$ of 2 natural numbers $x$ and $y$. The extension consists in
the addition of 2 variables $u$ and $v$, and statements which lead to the computation of the
least common multiple $lcm$ of $x$ and $y$. The $lcm$ and $gcd$ are related by the equation
\begin{verbatim}
  lcm(x,y) * gcd(x,y) = x*y
\end{verbatim}

As explained earlier, this and following examples will be formulated as command procedures and
are assumed to be embedded in module \verb|Pattern| defined in \ref{ch:1ex} which imports
modules \verb|Texts| and \verb|Oberon|.
\begin{verbatim}
  PROC gcdlcm*;
    VAR x, y, u, v: INT; S: Texts.Scanner;
  BEGIN Texts.OpenScanner(S, Oberon.Par.text,
                             Oberon.Par.pos):
    Texts.Scan(S); x := S.i;
    Texts.WriteString(W, " x =");
    Texts.WriteInt(W, x, 6);
    Texts.Scan(S); y := S.i;
    Texts.WriteString(W, " y =");
    Texts.WriteInt(W, y, 6);
    u := x; v := y;
    (*gcd(x,y) = gcd(x0,yO), x*v + y*u = 2*x0*y0*)
    WHILE x # y DO
      IF x > y THEN x := x-y; u := u+v
               ELSE y := y-x; v := v+u
      END
    END;
    Texts.WriteInt(W, x, 6);
    Texts.WriteInt(W, (u+v) DIV 2, 6);
    Texts.WriteLn(W);
    Texts.Append(Oberon.Log, W.buf)
  END gcdlcm.
\end{verbatim}
This example again shows the nesting of control structures. The repetition expressed by a while
statement includes a conditional structure expressed by an if statement, which in turn includes
2 statement sequences, each consisting of 2 assignments. This hierarchical structure is made
transparent by appropriate indentation of the "inner" parts.

Another example demonstrating a hierarchical structure computes the real number $x$ raised to
the power $i$, where $i$ is a non-negative integer.
\begin{verbatim}
  PROC Power*;
    VAR i: INT; x, z: REAL; S: Texts.Scanner;
  BEGIN Texts.OpenScanner(S, Oberon.Par.text,
                             Oberon.Par.pos);
    Texts.Scan(S);
    WHILE S.class = Texts.Real DO
      x := S.x; Texts.WriteString(W, " x =");
      Texts.WriteReal(W, x, 16); Texts.Scan(S);
      i := S.i; Texts.WriteString(W, " i =");
      Texts.WriteInt(W, i, 4);
      z= 1.0:
      (* z * x^i = x0^i0 *)
      WHILE i > 0 DO z := z*x; i := i-1 END;
      Texts.WriteReal(W, z, 16);
      Texts.WriteLn(W); Texts.Scan(S)
    END;
    Texts.Append(Oberon.Log, W.buf)
  END Power.
\end{verbatim}
Here we subject the computation to yet another repetition: Each time a result has been computed,
another value pair $x, i$ is requested. This outermost repetition is controlled by the relation
\verb|S.class = Texts.Real|, which indicates whether a real number $x$ had actually been read.
As an example, activation of the command
\begin{verbatim}
  M.Power 5.0 2   2.0 5   3.0 3 ~
\end{verbatim}
will generate the results \verb|25.0, 32.0, 27.0|.

The straight-forward computation of a power by repeated multiplication is, although obviously
correct, not the most economical. We now present a more sophisticated and more efficient solution.
It is based on the following consideration: The goal of the repetition is to reach the value $i = 0$.
This is done by successively reducing $i$, while maintaining the invariant $z * x^i = x_0^{i_0}$,
where $x_0$ and $i_0$ denote the initial values of $x$ and $i$. A faster algorithm therefore must
rely on decreasing $i$ in larger steps. The solution given here halves $i$. But this is only possible,
if $i$ is even. Hence, if $i$ is odd, it is 1st decremented by 1. Of course, each change of $i$ must
be accompanied by a corrective action on $z$ in order to maintain the invariant.
\begin{itemize}
  \item A detail: the subtraction of 1 from $i$ is not expressed by an explicit statement, because
    it is performed implicitly by the subsequent division by 2.
  \item 2 further details are noteworthy:
  \begin{enumerate}
    \item The function \verb|ODD(i)| is \verb|TRUE|, if $i$ is an odd number, \verb|FALSE| otherwise.
    \item $x$ and $z$ denote real values, as opposed to integer values. Hence, they can represent
      fractions, too.
  \end{enumerate}
\end{itemize}
\begin{verbatim}
  PROC Power*;
    VAR i: INT; x, z: REAL; S: Texts.Scanner;
  BEGIN Texts.OpenScanner(S, Oberon.Par.text,
                             Oberon.Par.pos);
    Texts.Scan(S);
    WHILE S.class = Texts.Real DO
      x := S.x; Texts.WriteString(W, " x =");
      Texts.WriteReal(W, x, 16): Texts.Scan(S);
      i := S.i; Texts.WriteString(W, " i=");
      Texts.WriteInt(W, i, 4);
      z := 1.0;
      WHILE i > 0 DO
        (* z* x^i = x0^i0 *)
        IF ODD(i) THEN z := z*x END;
        x := x*x; i := i DIV 2
      END;
      Texts.WriteReal(W, z, 16);
      Texts.WriteLn(VW); Texts.Scan(S)
    END;
    Texts.Append(Oberon.Log, W.buf)
  END Power.
\end{verbatim}

The next sample command has a structure that is almost identical to the preceding program. It
computes the logarithm of a real number $x$ whose value lies between 1 and 2. The invariant in
conjunction with the termination condition ($b=0$) implies the desired result $sum=\log_2{x}$.
\begin{verbatim}
  PROC Log2*;
    VAR x, a, b, sum: REAL; S: Texts.Scanner;
  BEGIN Texts.OpenScanner(S, Oberon.Par.text,
                             Oberon.Par.pos);
    Texts.Scan(S);
    WHILE S.class = Texts.Real DO
      x := S.x; Texts.WriteReal(W, x, 15); (*1<=x<2*)
      a :=x; b := 1.0; sum := 0.0;
      REPEAT (*log_2(x) = sum + b*log_2(a)*)
        a := a*a; b := 0.5*b;
        IF a >= 2.0 THEN
          sum := sum + b; a := 0.5*a
        END
      UNTIL b < 1.0E-7;
      Texts.WriteReal(W, sum, 16);
      Texts.WriteLn(W); Texts.Scan(S)
    END;
    Texts.Append(Oberon.Log, W.buf)
  END Log2.
\end{verbatim}

Normally, routines for computing standard mathematical functions need not be programmed in
detail, because they are available from a collection of modules similar to those for input
and output. Such a collection is, somewhat inappropriately, called a library. In the following
example, again exhibiting the use of a repetitive statement, we use routines for computing the
cosine and the exponential function from a library called Math and generate a table of values
for a damped oscillation. Typically, the available standard routines include the $sin$, $cos$,
$exp$, $ln$ (logarithm), $sqrt$ (square root), and the $arctan$ functions.
\begin{verbatim}
  PROC Oscillation*;
    CONST dx = 0.19634953; (*pi/16*)
    VAR i, n: INT; x, y, REAL; S: Texts.Scanner;
  BEGIN Texts.OpenScanner(S, Oberon.Par.text,
                             Oberon.Par.pos):
    Texts.Scan(S); n := S.i; Texts.WriteInt(W, n, 6);
    Texts.Scan(S); r := S.x; Texts.WriteReal(W, r, 15);
    Texts.WriteLn(W); i := 0; x := 0.0;
    REPEAT x := x + dx; i := i+1;
           y := Math.exp(-r*x) * Math.cos(x);
           Texts.WriteReal(W, x, 15);
	   Texts.WriteReal(W, y, 15);
	   Texts.WriteLn(W)
    UNTIL i >= n;
    Texts.Append(Oberon.Log, W.buf)
  END Oscillation.
\end{verbatim}
