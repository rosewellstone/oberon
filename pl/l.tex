\chapter{Decomposition into Modules}
The quality of a program has many aspects and is an elusive property. A user of a program may
judge it according to its efficiency, reliability, or convenience of user dialog. Whereas efficiency can
be expressed in terms of numbers, convenience of usage is rather a matter of personal judgement,
and all too often a program's usage is called convenient as long as it is conventional. An engineer
of a program may judge its quality according to its clarity and perspicuity, again rather elusive and
subjective properties. However, if a property cannot be expressed in terms of precise numbers, this
is no reason for classifying it as irrelevant. In fact, program clarity is enormously important, and to
demonstrate the correctness of a program is ultimately a matter of convincing a person that the
program is trustworthy. How can we approach this goal? After all, complicated tasks usually do
inherently require complex algorithms, and this implies a myriad of details. And the details are the
jungle in which the devil hides.

The only salvation lies in structure. A program must be decomposed into partitions which can be
considered one at a time without too much regard for the remaining parts. At the lowest level the
elements of the structure are statements, at the next level procedures, and at the highest level
modules. In parallel with program structuring proceeds the structuring of data into arrays, records,
etc. at the lower levels, and through association of variables with procedures and modules at the
higher levels. The essence of programming is finding the right (or at least an appropriate) structure,
and the experienced programmer is the person who has the intuition to find it at the stage of initial
conception instead of during a gradual process of improvements and modifications. Yet, the
programmer who has the courage to restructure when a better solution emerges is still much better
off than the one who resigns and elaborates a program on the basis of a clearly inadequate
structure, for this leads to those products that no one else, and ultimately not even the originator
himself can "understand".

Even if there does not exist a recipe to determine the optimal structure of a program, there have
emerged some criteria for guidance in the process of finding good and avoiding bad structure. A
basic rule is that decomposition should be such that the connections between partitions are simple
or "thin". A perhaps oversimplifying criterion for the thickness of a connection - also called interface
- between 2 parts is the number of items that take part in it. Specifically, the interface of 2
modules is sketched in terms of the module's import lists, and a measure for the interface's
thickness is the number of imported items. Hence, we must find a modularization which makes the
import lists short. Naturally, it is difficult to find an optimum, for, the import lists would be shortest,
i.e. they would disappear if the entire program were collapsed into a single module: a clearly
undesirable solution.

The distinctive property of the module as the largest structuring unit is its capability to hide details
and thereby to establish a new level of abstraction. This property is used in several forms; we can
distinguish between the following cases:
\begin{enumerate}
  \item The module separates 2 kinds of data representation and contains the collection of procedures
that perform the data conversion between the 2 levels. The typical example is a module for
conversion of numbers from their abstract, atomic representation into sequences of decimal digits
and vice-versa. Such modules contain no data of their own, they are typically packages of
procedures.

  \item The essence of a module is a set of data. It hides the details of the data representation by
granting access to these data through calls of its exported procedures only. An example of this
case is a module which contains a data set storing individual items organized in a way that items
are found quickly. Another is a module whose hidden data set is a disk store; it hides the peculiar
details necessary to operate the disk drive.

  \item The module exports a data type and exports its associated operations. Typically, in Oberon such
a module exports one or several types in opaque mode (sometimes these are also called private
types). It thereby hides the details of the data type's structure and also the details of the
operations. By hiding them, it is possible to guarantee the validity of postulated invariant
properties of each variable of such a private type. The difference to modules of class 2 is that
here variables of the private types are declared in the client modules, whereas in class 2 modules
the variable is itself hidden. Typical examples are the queue and stack types, and perhaps the
most successful such data abstraction is the sequential file, also known as a stream.
\end{enumerate}
This classification is not absolute. It cannot be, because all kinds share the common goal of hiding
details. Nevertheless, we can formulate a few rules that serve as guidelines in the design of modules:
\begin{enumerate}
  \item Keep the number of imported modules small.
  \item Keep the number of imported modules in definitions even smaller.
  \item Export of variables should be considered as the exception, and imported variables should be
    treated as "read-only" objects.
\end{enumerate}

We conclude this chapter with an example that essentially falls into category 3. Let our stated goal
be the design of a cross reference generator for Oberon programs. More precisely, the program's
purpose is to read a text and to generate (1) a listing of the text with added line numbers and (2) a
table of all encountered words (identifiers) in alphabetical order, each with a list of the numbers of
the lines in which the word occurs. Moreover, comments and strings are to be skipped (i.e. their
words are not to be listed), and Oberon key symbols are not to be listed either.

We quickly recognize the task as being divisible into the scanning of the source text (eliminating the
parts that are to be skipped and ignored), and the recording and subsequent tabulating of the
words. The first part is conveniently performed by the main program module, the latter by a
subsidiary module which hides the data set and makes it accessible through 2 procedures: Insert
(i.e. include a word) and List (i.e. generate the requested table). A 3rd module is used to generate
the representation of numbers as sequences of decimal digits. The 3 principal modules involved
are called \verb|XREF|, \verb|TableHandler|, and \verb|Texts|.

We begin by presenting the main program \verb|XREF| that scans the source text. A binary search is used
to recognize key words. The data set is of the type \verb|Table|, imported from the \verb|TableHandler|.
\begin{verbatim}
  DEFINITION TableHandler;
    CONST WordLength; TYPE Word;
    PROC Init (VAR w: Word);
    (*enter pair s,In in structure w *)
    PROC Insert(VAR s: ARRAY OF CHAR; In: INT;
                VAR w: Word);
    PROC List(w: Word)
  END TableHandler.
 
  MODULE XREF;
    IMPORT Texts, Oberon, TH := TableHandler;
    CONST N = 32; L = 8; (*keywords #, maxlen*)
    VAR ln: INT; (*current line number")
        Tab: TableHandler.Word; W: Texts.Writer;
        key: ARRAY N, L OF CHAR; (*table of keys*)
 
    PROC heading;
    BEGIN INC(ln); Texts.WriteInt(W, ln, 5);
                   Texts.Write(W, 20X)
    END heading;
 
    PROC Scan*; (*commana’*)
      VAR beg, end, time: LONGINT;
          k, m, l, r: INT; c: CHAR;
          id: ARRAY TH.WordLength OF CHAR;
          T: Texts. Text; R: Texts.Reader;
      PROC copy;
      BEGIN Texts.Write(W, c); Texts.Read(R, c);
      END copy;
    BEGIN TableHandler.Init(Tab);
      Oberon.GetSelection(T, beg, end, time);
      IF time >= 0 THEN ln := 0; heading;
        Texts.OpenReader(R, T, beg);
        Texts.Read(R, c);
        REPEAT
          IF (CAP(c)>="A") & (CAP(c)<="Z") THEN (*word*)
            k := 0;
            REPEAT id[k] := c; INC(k); copy
            UNTIL ~(("A"<=CAP(c)) & (CAP(c)<="Z")
                  OR ("a" <= c) & (c <= "z"));
            id[k] := OX;
            l := 0; r := N; (*binary search for key*)
            REPEAT m := (l+r) DIV 2;
              IF key[m] < id THEN l := m + 1
                             ELSE r := m END;
            UNTIL l >= r;
            IF (r = N) OR (id # key[r]) THEN
              TableHandler.Insert(id, ln, Tab)
            END
          ELSIF (c >= "0") & (c <= "9") THEN (*number*)
            REPEAT copy UNTIL ~("0"<=c) & (c<="9")
          ELSIF c = "(" THEN copy;
            IF c="*" THEN (*skip comment*)
              REPEAT
                REPEAT IF c=0DX THEN heading END; copy
                UNTIL c = "*"; copy
              UNTIL c = ")"; copy
            END
          ELSIF c = 22X THEN (*string*)
            REPEAT copy UNTIL c = '"'; copy
          ELSIF c = ODX THEN (*end of line*)
            copy; heading
          ELSE copy
          END
        UNTIL R.eot; Texts.WriteLn(W);
        Texts.Append(Oberon.Log, W.buf);
        TableHandler.List(Tab)
      END
    END Scan;

  BEGIN Texts.OpenWriter(W);
    key[ 0] := "AND"   ; key[ 1] := "ARRAY"  ;
    key[ 2] := "BEGIN" ; key[ 3] := "BOOL"   ;
    key[ 4] := "BY"    ; key[ 5] := "CASE"   ;
    key[ 6] := "CONST" ; key[ 7] := "DIV"    ;
    key[ 8] := "DO"    ; key[ 9] := "ELSE"   ;
    key[10] := "ELSIF" ; key[11] := "END"    ;
    key[12] := "FOR"   ; key[13] := "IF"     ;
    key[14] := "IMPORT"; key[15] := "IN"     ;
    key[16] := "MOD"   ; key[17] := "MODULE" ;
    key[18] := "NOT"   ; key[19] := "OF"     ;
    key[20] := "OR"    ; key[21] := "POINTER";
    key[22] := "PROC"  ; key[23] := "RECORD" ;
    key[24] := "REPEAT"; key[25] := "RETURN" ;
    key[26] := "THEN"  ; key[27] := "TO"     ;
    key[28] := "TYPE"  ; key[29] := "UNTIL"  ;
    key[30] := "VAR"   ; key[31] := "WHILE"
  END XREE.
\end{verbatim}
Next we present the table handler. As seen from its definition part, it exports the private
type \verb|Table| and its operations \verb|Insert| and \verb|List|. Notably the structure
of the tables, and thereby also the access and search algorithms remain hidden. The 2 most
likely choices are the organizations of binary trees and of a hash table. Here we opt for
the former. The example is therefore a further illustration of the use of pointers and
dynamic data structures. The module contains a procedure to search and insert a tree element,
and a procedure that traverses the tree for the required tabulation (see also \ref{ch:ptr}
on dynamic data structures). Each tree node is a record with fields for the key, the left
and right descendants, and (the head of) a list of the line numbers. Another form of
representation might be chosen, for example a balanced tree, and such a new implementaion
might be provided without clients being aware of the change, because modules can be compiled
separately. This is the key of modularization and constructing large systems. It requires,
however, that the modules’ interfaces are wisely chosen.
\begin{verbatim}
  MODULE TableHandler;
    IMPORT Texts, Oberon;
    CONST WordLength* = 32;
    TYPE Item = POINTER TO RECORD
                  num: INT; next: Item
                END;
         Word* = POINTER TO RECORD
                   key: ARRAY WordLength OF CHAR;
                   first: Item; (“list head*)
                   left, right: Word
                 END;
    VAR W: Texts.Writer;
 
    PROC Init*(VAR w: Word);
    BEGIN w := NIL
    END Init;
 
    (“search node with name s*)
    PROC Insert*(VAR s: ARRAY OF CHAR; ln: INT; VAR w: Word);
      VAR t: Item;
    BEGIN
      IF w = NIL THEN (*insert new word*)
        NEW(t); t.num := ln; t.next := NIL;
        NEW(w); w.first := t; w.left  := NIL;
        COPY(s, w.key);       w.right := NIL;
      ELSIF s < w.key THEN Insert(s, ln, w.left)
      ELSIF s > w.key THEN Insert(s, ln, w.right)
      ELSE NEW(t); t.num := ln;
        t.next := w.first; w.first := t
      END
    END Insert;
 
    PROC write(w: Word);
      VAR t: Item;
    BEGIN Texts.WriteString(W, w.key);
      t := w.first;
      REPEAT Texts.VWriteInt(W, t.num, 5); t := t.next
      UNTIL t = NIL; Texts.WriteLn(W)
    END write;
 
    PROC List*(w: Word);
    BEGIN IF w # NIL THEN List(w.left); write(w);
                          List(w.right) END;
      Texts.Append(Oberon.Log, W.buf)
    END List;
 
  BEGIN Texts.OpenWriter(W)
  END TableHandler.
\end{verbatim}
